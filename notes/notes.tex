\documentclass[a4paper,twoside]{book}
\usepackage{geometry}
\usepackage[notheorems]{beamerarticle}



\mode<presentation>{
  % \useinnertheme{rectangles}
  %\useoutertheme{infolines}
  % \usecolortheme{crane}
  % \usecolortheme{rose}
}
\input{../preamble}

%\includeonly{database,privacy}
%\includeonly{experiment-design}
%\includeonly{networks}
\title{Machine learning in science and society}
\subtitle{From automated science to beneficial artificial intelligence}
\author[C. Dimitrakakis]{Christos Dimitrakakis}
\begin{document}

\maketitle
\tableofcontents

\chapter{Introduction}
\include{ml-intro}

\chapter{Simple decision problems}

\only<article>{This chapter deals with simple decision problems, whereby a decision maker (DM) makes a simple choice among many. In some of this problems the DM has to make a decision after first observing some side-information. Then the DM uses a \emph{decision rule} to assign a probability to each possible decision for each possible side-information. However, designing the decision rule is not trivial, as it relies on previously collected data. A higher-level decision includes choosing the decision rule itself. The problems of classification and regression fall within this framework. While most steps in the process can be automated and formalised, a lot of decisions are actual design choices made by humans. This creates scope for errors and misinterpretation of results.

In this chapter, we shall formalise all these simple decision problems from the point of view of statistical decision theory. The first question is, given a real world application, what type of decision problem does it map to? Then, what kind of machine learning algorithms can we use to solve it? What are the underlying assumptions and how valid are our conclusions? 
}

\section{Nearest neighbours}
\only<presentation>{
  \begin{frame}
    \tableofcontents[ 
    currentsection, 
    hideothersubsections, 
    sectionstyle=show/shaded
    ] 
  \end{frame}
}


\begin{frame}
  \frametitle{Discriminating between diseases}
  \input{../figures/KNN3.tikz}
\end{frame}

\only<article>{
  Let's tackle the problem of discriminating between different
  disease vectors. Ideally, we'd like to have a simple test that
  tells us what ails us. One kind of test is mass spectrometry. This
  graph shows spectrometry results for two types of bacteria. There
  is plenty of variation within each type, both due to measurement
  error and due to changes in the bacterial strains. Here, we plot
  the average and maximum energies measured for about 100 different
  examples from each strain.
}


\begin{frame}
  \frametitle{Nearest neighbour: the hidden secret of machine learning}
  \input{../figures/separation1.tikz}
\end{frame}


\only<article>{ Now, is it possible to identify an unknown strain
  based on this data? Actually, this is possible. Sometimes, very
  simple algorithms work very well. One of the simplest one involves
  just measuring the distance between the decsription of a new unknown
  strain and known ones. In this visualisation, I projected the
  1300-dimensional data into a 2-dimensional space. Here you can
  clearly see that it is possible to separate the two strains. We can
  use the distance to examples VVT and BUT in order to decide the type
  of an unknown strain.  }

\begin{frame}
  \frametitle{Comparing spectral data}
  \only<1>{\input{../figures/difference1.tikz}}
  \only<presentation>{\only<2>{\input{../figures/difference2.tikz}}}
\end{frame}

\only<article>{
  The choice of distance in this kind of algorithm is important,
  particularly for very high dimensions. For something like a
  spectrogram, one idea is look at the total area of the difference
  between two spectral lines. 
}

\begin{frame}
  \frametitle{The nearest neighbour algorithm}
  \only<article>{The nearest neighbour algorithm for classification (Alg.~\ref{alg:kNN-classify}) does not include any complicated learning. Given a training dataset $D$, it returns a classification decision for any new point $x$ by simply comparing it to its closest $k$ neighbours in the dataset. It then estimates the probability $p_y$ of each class $y$ by calculating the average number of times the neighbours take the class $y$.
  }
  \begin{algorithm}[H]
    \begin{algorithmic}[1]
      \State \textbf{Input} Data $D = \{(x_1, y_1), \ldots, (x_T, y_T)\}$, $k \geq 1$,  $d : \CX \times \CX \to \Reals_+$, new point $x \in \CX$
      \State $D = \texttt{Sort}(D, d)$ \% \textsf{ Sort $D$ so that $d(x, x_i) \leq d(x, x_{i+1})$}.
      \State $p_y = \sum_{i=1}^k \ind{y_i = y} / k$ for $y \in \CY$.
      \State \textbf{Return} $\vp \defn (p_1, \ldots, p_k)$
    \end{algorithmic}
    \caption{\KNN{} Classify}
    \label{alg:kNN-classify}
  \end{algorithm}
  \begin{alertblock}{Algorithm parameters}
    \only<article>{In order to use the algorithm, we must specify some parameters, namely.}
    \begin{itemize}
    \item Neighbourhood $k \geq 1$. \only<article>{The number of neighbours to consider.}
    \item Distance $d : \CX \times \CX \to \Reals_+$. \only<article>{The function we use to determine what is a neighbour.}
    \end{itemize}
  \end{alertblock}
  \only<presentation>{
    What does the algorithm output when $k = T$?
  }
\end{frame}


\begin{frame}
  \begin{figure}[H]
    \centering
    \includegraphics[width=\fwidth]{../figures/fix_evelyn2}
    \includegraphics[width=\fwidth]{../figures/Hodges}
    \caption{The nearest neighbours algorithm was introduced by \citet{fix1951discriminatory}, who also proved consistency properties.}
  \end{figure}
\end{frame}

\begin{frame}
  \frametitle{Nearest neighbour: What type is the new bacterium?}
  \input{../figures/separation2.tikz}
  \only<presentation>{
    \only<2>{\newline \Large \alert{What if it a \textbf{completely different strain}?}}
  }
  \only<article>{Given that the $+$ points represent the BUT type, and the $\times$ points the VVJ type, what type of bacterium could the circle point be?}
\end{frame}

\begin{frame}
  \frametitle{Separating the model from the classification policy}
  \begin{itemize}
  \item The \KNN{} algorithm returns a model giving class probabilities for new data points.
  \item \only<article>{It is up to us to decide how to use this model to decide upon a given class. A typical decision making rule can be in the form of a policy $\pol$ that depends on what the model says. However, the simplest decision rule is to take the most likely class:}
    \only<presentation>{Deciding a class given the model}
    \[
      \pol(a \mid x) = \ind{p_a \geq p_y \forall y}, \qquad \vp = \KNN(D, k, d, x)
    \]
  \end{itemize}
\end{frame}

\only<presentation>{
  \begin{frame}
    \frametitle{Hands on with Python console}
    \hyperlink{../src/decision-problems/knn-classify.py}{\beamerbutton{KNN example}}
    %% Use knn-classify to simply demonstrate a kNN classifier.
  \end{frame}
}

\begin{frame}
  \frametitle{Discussion: Shortcomings of $k$-nearest neighbour}
  \begin{itemize}
  \item Choice of $k$ \only<article>{The larger $k$ is, the more data you take into account when making your decision. This means that you expect your classes to be more spread out.} 
  \item Choice of metric $d$. \only<article>{The metric $d$ encodes prior information you have about the structure of the data.}
  \item Representation of uncertainty. \only<article>{The predictions of kNN models are simply based on distances and counting. This might not be a very good way to represent uncertainty about class label. In particular, label probabilities should be more uncertain when we have less data.}
  \item Scaling with large amounts of data. \only<article>{A naive implementation of kNN requires the algorithm to shift through all the training data to find the $k$ nearest neighbours, suggesting a superlinear computation time. However, advanced data structures such as Cover Trees (or even KD-trees in low dimensional spaces) can be used to find the neighbours in polylog time.}
    \item Meaning of label probabilities.  \only<article>{It is best to think of \KNN{} as a \alert{model} for predicting the class of a new example from a finite set of existing classes. The model itself might be incorrect, but this should nevertheless be OK for our purposes. In particular, we might later use the model in order to derive classification rules.}
  \end{itemize}


\end{frame}

\begin{frame}
  \frametitle{Learning outcomes}
  \begin{block}{Understanding}
    \begin{itemize}
    \item How kNN works
    \item The effect of hyperameters $k, d$ for nearest neighbour.
    \item The use of kNN to classify new data.
    \end{itemize}
  \end{block}
  
  \begin{block}{Skills}
    \begin{itemize}
    \item Use a standard kNN class in python
    \item Optimise kNN hyperparameters in an unbiased manner.
    \item Calculate probabilities of class labels using kNN.
    \end{itemize}
  \end{block}

  \begin{block}{Reflection}
    \begin{itemize}
    \item When is kNN a good model?
    \item How can we deal with large amounts of data?
    \item How can we best represent uncertainty?
    \end{itemize}
  \end{block}
  
\end{frame}


%%% Local Variables:
%%% mode: latex
%%% TeX-master: "notes.tex"
%%% End:
  % knn, reproducability and bootstrapping
\section{Reproducibility}
\only<presentation>{
  \begin{frame}
    \tableofcontents[ 
    currentsection, 
    hideothersubsections, 
    sectionstyle=show/shaded
    ] 
  \end{frame}
}
\begin{frame}
  \only<article>{ One of the main problems in science is
    reproducibility: when we are trying to draw conclusions from one
    specific data set, it is easy to make a mistake. For that reason,
    the scientific process requires us to use our conclusions to make
    testable predictions, and then test those predictions with new
    experiments. These new experiments should bear out the results of
    the previous experiments.  In more detail, reproducibility can be
    thought of as two different aspects of answering the question ``can this research be replicated?''}


  \begin{block}{Computational reproducibility: Can the study be repeated?}
    Can we, from the available information and data, exactly reproduce the reported methods and results?

    \only<article>{This is something that is useful to be able to even to the original authors of a study. The standard method for achieving this is using version control tools so that the exact version of algorithms, text and data used to write up the study is appropriately labelled. Ideally, any other researcher should be able to run a single script to reproduce all of the study and its computations. The following tools are going to be used in this course:}

    \begin{itemize}
    \item \texttt{jupyter} notebooks \only<article>{for interactive note taking.}
    \item \texttt{svn}, \texttt{git} or \texttt{mercurial} version control systems \only<article>{for tracking versions, changes and collaborating with others.}
    \end{itemize}
  \end{block}

  \begin{block}{Scientific reproducibility: Is the conclusion correct?}
    Can we, from the available information and a \alert{new} set of data, reproduce the conclusions of the original study?

     \only<article>{Here followup research may involve using exactly the same methods. In AI research would mean for example testing whether an algorithm is really performing as well as it is claimed, by testing it in new problems. This can involve a re-implementation. In more general scientific research, it might be the case that the methodology proposed by an original study is flawed, and so a better method should be used to draw better conclusions. Or it might simply be that more data is needed.}
  \end{block}

  When publishing results about a \alert{new method}, computational reproducibility is essential for scientific reproducibility.
\end{frame}


\begin{frame}

  \includegraphics[width=\textwidth]{../figures/2016-election}
  \only<article>{A simple example is the 2016 election. While we can make models about people's opinions regarding candidates in order to predict voting totals, the test of these models comes in the actual election. Unfortunately the only way we have of tuning our models is on previous elections, which are not that frequent, and on the results of previous polls. In addition, predicting the winner of an election is slightly different from predicting how many people are prepared to vote for them across the country. This, together with other factors such as shifting opinions, motivation and how close the sampling distribution is to the voting distribution have a significant effect on accuracy.}
  
\end{frame}
\begin{frame}
  \only<article>{The same thing can be done in when dealing purely
    with data, by making sure we use some of the data as input to the
    algorithm, and other data to measure the quality of the algorithm
    itself. In the following, we assume we have some algorithm
    $\alg : \Datasets \to \Pol$, where $\Datasets$ is the universe of
    possible input data and $\Pol$ the possible outputs, e.g. all
    possible classification policies. We also assume the existence of
    some quality measure $U$. How should we measure the quality of our
    algorithmic choices? 
    
    Take classification as an example. For a given training set, simply memorising all the labels of each example gives us perfect performance on the training set. Intuitively, this is not a good measure of performance, as we'd probably get poor results on a freshly sampled set. We can think of the training data as input to an algorithm, and the resulting classifier as the algorithm output. The evaluation function also requires some data in order to measure the performance of the policy. This can be expressed into the following principle.
  }
  \begin{alertblock}{The principle of independent evaluation}    
    Data used for estimation cannot be used for evaluation.
  \end{alertblock}
  \only<article>{This applies both to computer-implemented and human-implemented algorithms.}
\end{frame}


\begin{frame}
  \begin{figure}[H]
    \begin{center}
      \begin{tikzpicture}[line width=2pt]
        \onslide<1->{
          \node[select,label=above:Data Collection] at (0,2) (experiment) {$\chi$};
        }
        \onslide<2->{
          \node[RV,label=above:Training] at (4,2) (training) {$\Training$};
          \draw[blue,->] (experiment) -- (training);
        }
        \onslide<3->{
          \node[select,label=below:{Algorithm, hyperparameters}] at (0,0) (alg) {$\alg$};
        }
        \onslide<4->{
          \node[RV,label=below:Classifier] at (4,0) (pol) {$\pol$};
          \draw[red,->,dashed] (experiment) -- (pol);
          \draw[red,->] (alg) -- (pol);
          \draw[red,->] (training) -- (pol);
        }
        \onslide<5->{
          \node[RV,label=above:Holdout] at (8,2) (holdout) {$\Holdout$};
        }
        \onslide<6->{
          \draw[blue,->] (experiment) to [bend left=45] (holdout);
        }
        \onslide<7->{
          \node[utility,label=below:Measurement] at (8,0) (util) {$\util$};
          \draw[red,->] (pol) -- (util);
          \draw[red,->] (holdout) -- (util);
        }
      \end{tikzpicture}
    \end{center}
    \caption{The decision process in classification.}
  \end{figure}
  \only<article>{One can think of the decision process in classification as follows. First, we decide to collect some data according to some experimental protocol $\chi$. We also decide to use some algorithm (with associated hyperparameters) $\alg$ together with data $\Training$ we will obtain from our data collection in order to obtain a classification policy $\pol$. Typically, we need to measure the quality of a policy according to how well it classifies on unknown data. This is because our policy has been generated using $\Training$, and so any measurement of its quality is going to be biased.

    For classification problems, there is a natural metric $U$ to measure. The classification accuracy of the classifier. If the classification decisions are stochastic, then the classifier assigns probability $\pol(a \mid x)$ to each possible label $a$, and our utility is simply the identity function $U(a, y) \defn \ind{a = y}$. 
  }
  \uncover<8->{
    \begin{block}{Classification accuracy}
      \only<8>{
      \[
      \E_\chi[\util(\pol)] = \sum_{x,y} \underset{\textrm{Data probability}}{\underbrace{\Pr_\chi(x, y)}} \overset{\textrm{Decision probability}}{\overbrace{\pol(a = y \mid x)}}
      \]
      }
      \only<article>{The classification accuracy of policy $\pol$ under $\chi$ is the expected number of times the policy decides $\pol$ chooses the correct class. However, when approximating $\chi$ with a sample $\Holdout$, we instead obtain the empirical estimate:}
      \only<9>{
      \[
      \E_{\Holdout} \util(\pol) = \sum_{(x,y) \in \Holdout} \pol(a = y \mid x) / |\Holdout|.
      \]
      }
    \end{block}
  }
  \only<article>{Of course, there is no reason to limit ourselves to the identity function. The utility could very well be such that some errors are penalised more than other errors. Consider for example an intrusion detection scenario: it is probably more important to correctly classify intrusions. }
\end{frame}

\subsection{The human as an algorithm}
\begin{frame}
  \frametitle{The human as an algorithm.}
  \only<article>{The same way with which an algorithm creates a model from some prior assumptions and data, so can a human select an algorithm and associated hyperparamters by executing an algorithm herself. This involves trying different algorithms and hyperparametrs on the same training data $\Training$ and then measuring their performance in the holdout set $\Holdout$.}
  \begin{figure}[H]
  \centering
  \begin{tikzpicture}[line width=2pt]
    \node[select,label=above:Data Collection] at (0,2) (experiment) {$\chi$};
    \node[RV,label=above:Training] at (4,2) (training) {$\Training$};
    \node[RV,label=above:Holdout] at (8,2) (holdout) {$\Holdout$};
     \draw[blue,->] (experiment) -- (training);
    \draw[blue,->] (experiment) to [bend left=45] (holdout);
    \node<2->[select,label=below:{Algorithm, hyperparameters}] at (0,0) (alg) {$\alg_1$};     
    \node<3->[RV,label=below:Classifier] at (4,0) (pol) {$\pol_1$};
    \draw<3->[red,->] (alg) -- (pol);
    \draw<3->[red,->] (training) -- (pol);
    \node<4->[utility,label=below:Measurement] at (8,0) (util) {$\util_1$};
    \draw<4->[red,->] (pol) -- (util);
    \draw<4->[red,->] (holdout) -- (util);
    \node<5->[select,label=below:{Algorithm, hyperparameters}] at (0,-2) (alg2) {$\alg_2$};
    \node<6->[RV,label=below:Classifier] at (4,-2) (pol2) {$\pol_2$};
    \node<7->[utility,label=below:Measurement] at (8,-2) (util2) {$\util_2$};
    \draw<6->[red,->] (alg2) -- (pol2);
    \draw<6->[red,->] (training) to [bend left] (pol2);
    \draw<7->[red,->] (pol2) -- (util2);
    \draw<7->[red,->] (holdout) to [bend right] (util2);
  \end{tikzpicture}
    \caption{Selecting algorithms and hyperparameters through holdouts}
    \label{fig:human-as-algorithm}
  \end{figure}

\end{frame}

\begin{frame}
  \frametitle{Holdout sets}
  \only<article>{To summarise, holdout sets are used in order to be able to evaluate the performance of specific algorithms, or hyparameter selection.}
  \begin{itemize}
  \item Original data $\Data$, e.g. $\Data = (x_1, \ldots, x_T)$.
  \item Training data $\Training \subset \Data$, e.g. $\Training = x_1, \ldots, x_n$, $n < T$.
  \item Holdout data $\Holdout = D \setminus \Training$, used to measure the quality of the result.
  \item Algorithm $\alg$ with hyperparametrs $\hyperparam$.
  \item Get algorithm output $\pol = \alg(\Training, \hyperparam)$.
  \item Calculate quality of output $U(\pol, \Holdout)$
  \end{itemize}
  \only<article>{
    We start with some original data $\Data$, e.g. $\Data = (x_1, \ldots, x_T)$. We then split this into a training data set $\Training \subset \Data$, e.g. $\Training = x_1, \ldots, x_n$, $n < T$ and holdout dataset $\Holdout = D \setminus \Training$. This is used to measure the quality of selected algorithms $\alg$ and hyperparameters $\hyperparam$. We run an algorithm/hyperparameter combination on the training data and obtain a result $\pol = \alg(\Training, \hyperparam)$.    \footnote{As typically algorithms are maximising the quality metric on the training data, 
    \[
    \alg(\Training) = \argmax_y U(y, \Training)
    \]
    we typically obtain a biased estimate, which depends both on the algorithm itself and the training data. For \KNN{} in particular, when we measure accuracy on the training data, we can nearly always obtain near-perfect accuracy, but not always perfect. Can you explain why?}
 We then calculate the quality of the output $U(\pol, \Holdout)$ on the holdout set.
    Unfortunately, the combination that appears the best due to the holdout result may look inferior in a fresh sample. Following the principle of ``data used for evaluation cannot be used for estimation'', we must measure performance on another sample.
This ensures that we are not biased in our decision about what is the best algorithm.
  }
  \begin{block}{Holdout and test sets for unbiased algorithm comparison}
    \only<article>{
      Consider the problem of comparing a number of different algorithms in $\Alg$. Each algorithm $\alg$ has a different set of hyperparameters $\Hyperparam_\alg$. The problem is to choose the best parameters for each algorithm, and then to test them independently. A simple meta-algorithm for doing this is based on the use of a \emph{holdout} set for choosing hyperparameters for each algorithm, and a \emph{test} set to measure algorithmic performance.}
    \begin{algorithm}[H]
      \begin{algorithmic}
        \State Partition data into $\Training, \Holdout, \Testing$.
        \For {$\alg \in \Alg$} \For
        {$\hyperparam \in \Hyperparam_\alg$} \State
        $\pol_{\hyperparam, \alg} = \alg(\Training, \hyperparam)$.
        \EndFor
        \State Get $\pol^*_{\alg}$ maximising
        $\util(\pol_{\hyperparam, \alg}, \Holdout)$.  \State
        $u_\alg = \util(\pol^*_\alg, \Testing)$.
        \EndFor
        \State $\alg^* = \argmax_\alg u_\alg$.
      \end{algorithmic}
      \caption{Unbiased adaptive evaluation through data partitioning}
    \end{algorithm}
  \end{block}
\end{frame}

\begin{frame}
  \frametitle{Final performance measurement}
  \only<article>{When comparing many algorithms, where we must select a hyperparameter for each one, then we can use one dataset as input to the algorithms, and another for selecting hyperparameters. That means that we must use another dataset to measure performance. This is called the testing set. Figure~\ref{fig:final-measurement} illustrates this.}
  \begin{figure}[H]
    \centering
  \begin{tikzpicture}[line width=2pt]
    \node[select,label=above:Data Collection] at (0,0) (experiment) {$\chi$};
    \node[RV,label=above:Training] at (2,0) (training) {$\Training$};
    \node[RV,label=above:Holdout] at (4,0) (holdout) {$\Holdout$};
    \node[RV,label=above:Testing] at (6,0) (testing) {$\Testing$};
    \draw[blue,->] (experiment) -- (training);
    \draw[blue,->] (experiment) to [bend left=15] (holdout);
    \draw[blue,->] (experiment) to [bend left=45] (testing);
    \node[select,label=below:{human}] at (0,-2) (human) {$\eta$};
    \node[RV,label=below:Algorithm 1] at (2,-2) (alg1) {$\alg_1$};
    \node[RV,label=below:Classifier 1] at (4,-2) (pol1) {$\pol_1$};
    \node[RV,label=below:Result 1] at (6,-2) (util1) {$\util^*_1$};
    \node[RV,label=below:Algorithm 2] at (2,-4) (alg2) {$\alg_2$};
    \node[RV,label=below:Classifier 2] at (4,-4) (pol2) {$\pol_2$};
    \node[RV,label=below:Result 2] at (6,-4) (util2) {$\util^*_2$};
    \draw[->,red] (human) -- (alg1);
    \draw[->,red] (alg1) -- (pol1);
    \draw[->,red] (pol1) -- (util1);
    \draw[->,red] (human) -- (alg2);
    \draw[->,red] (alg2) -- (pol2);
    \draw[->,red] (pol2) -- (util2);
    \draw[->,red] (training) -- (pol1);
    \draw[->,red] (holdout) -- (pol1);
    \draw[->,red] (testing) -- (util1);
    \draw[->,red] (training) -- (pol2);
    \draw[->,red] (holdout) to [bend left = 45] (pol2);
    \draw[->,red] (testing) to [bend left = 45] (util2);
  \end{tikzpicture}
  \only<article>{
    \caption{Simplified dependency graph for selecting hyperparameters for different algorithms, and comparing them on an independent test set. For the $i$-th algorithm, the classifier model is }
  }
  \label{fig:final-measurement}
\end{figure}
\end{frame}

\subsection{Algorithmic sensitivity}
\only<article>{The algorithm's output does have a dependence on its input, obviously. So, how sensitive is the algorithm to the input?}
\begin{frame}
  \frametitle{Independent data sets}
  \only<article>{One simple idea is to just collect independent datasets and see how the output of the algorithm changes when the data changes. However, this is quite expensive, as it not might be easy to collect data in the first place.}
  \begin{figure}[H]
    \centering
    \begin{tikzpicture}[line width=2pt]
      \node[select,label=above:Experiment] at (0,0) (experiment) {$\chi$};
      \node[select,label=below:{Algorithm}] at (8,0) (alg) {$\alg$};
      \node[RV,label=below:1st sample] at (4,0) (sample1) {$D_1$};
      \node[RV,label=below:1st Result] at (6,0) (pol1) {$\pol_1$};
      \node<2>[RV,label=above:2nd Sample] at (4,2) (sample2) {$D_2$};
      \node<2>[RV,label=above:2nd Result] at (6,2) (pol2) {$\pol_2$};
      \draw[blue,->] (experiment) -- (sample1);
      \draw[red,->] (alg) -- (pol1);
      \draw[red,->] (sample1) -- (pol1);
      \draw<2>[blue,->] (experiment) -- (sample2);
      \draw<2>[red,->] (alg) -- (pol2);
      \draw<2>[red,->] (sample2) -- (pol2);
    \end{tikzpicture}
    \caption{Multiple samples}
    \label{fig:multiple-samples}
  \end{figure}
\end{frame}
\begin{frame}
  \frametitle{Bootstrap samples}
  \only<article>{A more efficient idea is to only collect one dataset, but then use it to generate more datasets. The simplest way to do that is by sampling with replacement from the original dataset, new datasets of the same size as the original. Then the original dataset is sufficiently large, this is approximately the same as sampling independent datasets. As usual, we can evaluate our algorithm on an independent data set.}
  \begin{figure}[H]
  \centering
  \begin{tikzpicture}[line width=2pt]
    \node[select,label=above:Experiment] at (0,0) (experiment) {$\chi$};
    \node[RV,label=below:training] at (2,0) (training) {$\Training$};
    \draw[blue,->] (experiment) -- (training);
    \node[select,label=below:{Algorithm}] at (8,0) (alg) {$\alg$};
    \node[RV,label=below:1st sample] at (4,0) (sample1) {$D_1$};
    \node[RV,label=below:1st Result] at (6,0) (pol1) {$\pol_1$};
    \node[RV,label=above:2nd Sample] at (4,2) (sample2) {$D_2$};
    \node[RV,label=above:2nd Result] at (6,2) (pol2) {$\pol_2$};
    \draw[red,->] (alg) -- (pol1);
    \draw[red,->] (sample1) -- (pol1);
    \draw[red,->] (training) -- (sample1);
    \draw[red,->] (training) -- (sample2);
    \draw[red,->] (alg) -- (pol2);
    \draw[red,->] (sample2) -- (pol2);
  \end{tikzpicture}
  \caption{Bootstrap replicates of a single sample}
  \label{fig:bootstrap}
\end{figure}
\end{frame}



\begin{frame}
  \frametitle{Bootstrapping}
  Bootstrapping is a general technique that can be used to:
  \begin{itemize}
  \item Estimate the sensitivity of $\alg$ to the data $x$.
  \item Obtain a distribution of estimates $\pol$ from $\alg$ and the data $x$.
  \item When estimating the performance of an algorithm on a small dataset $\Testing$, use bootstrap samples of $\Testing$. \only<article>{This allows us to take into account the inherent uncertainty in measured performance. It is very useful to use bootstrapping with pairwise comparisons.}
  \end{itemize}
  \begin{block}{Bootstrapping}
    \begin{enumerate}
    \item \textbf{Input} Training data $\Data$, number of samples $k$.
    \item \textbf{For} $i = 1, \ldots, k$
    \item \quad $\Data^{(i)} = \textrm{Bootstrap}(\Data)$
    \item \textbf{return} $\cset{\Data^{(i)}}{i = 1, \ldots, k}$.
    \end{enumerate}
    where  $\textrm{Bootstrap}(\Data)$ samples with replacement $|\Data|$ points from $\Training$.
  \end{block}

\only<article>{
  In more detail, remember that even though the test score is an \emph{independent} measurement of an algorithm's performance, it is \emph{not} the actual expected performance. At best, it's an unbiased estimate of performance. Hence, we'd like to have some way to calculate a likely performance range from the test data. Bootstrapping can help: by taking multiple samples of the test set and calculating performance on each one, we obtain an empirical distribution of scores.

Secondly, we can use it to tell us something about the sensitivity of our algorithm. In particular, by taking multiple samples from the training data, we can end up with multiple models. If the models are only slightly different, then the algorithm is more stable and we can be more confident in its predictions. 

Finally, bagging also allows us to generate probabilistic predictions from deterministic classification algorithms, by simply averaging predictions from multiple bootstrapped predictors. This is called \emph{bagging predictors}\cite{breiman96bagging}.}
\end{frame}


\begin{frame}
  \frametitle{Cross-validation}
  \only<article>{While we typically use a single training, hold-out and test set, it might be useful to do this multiple times in order to obtain more robust performance estimates. In the simplest case, cross-validation can be used to obtain multiple training and hold-out sets from a single dataset. This works by simply partitioning the data in $k$ \emph{folds} and then using one of the folds as a holdout and the remaining $k - 1$ as training data. This is repeated $k$ times. When $k$ is the same size as the original training data, then the method is called \emph{leave-one-out cross-validation.}}
\begin{block}{$k$-fold Cross-Validation}
    \begin{enumerate}
    \item \textbf{Input} Training data $\Training$, number of folds $k$, algorithm $\alg$, measurement function $\util$
    \item Create the partition $\Data^{(1)} \ldots, \Data^{(k)}$ so that $\bigcup_{i=1}^k \Data^{(k)} = \Data$.
    \item Define $\Training^{(i)} = \Data \setminus \Data^{(i)}$
    \item \quad $\pol_i = \alg(\Training^{(i)})$
    \item \textbf{For} $i = 1, \ldots, k$:
    \item \quad $\pol_i = \alg(\Data^{(i)})$
    \item \quad $u_i = \util(\pol_i)$
    \item \textbf{return} $\{y_1, \ldots, y_i\}$.
    \end{enumerate}
  \end{block}
\end{frame}
\begin{frame}
\frametitle{Independent replication}
\only<article>{The gold standard for reproducibility is independent replication. Simply have another team try and reproduce the results you obtained, using completely new data. If the replication is successful, then you can be pretty sure there was no flaw in your original analysis.}
\begin{block}{Replication study}
  \begin{enumerate}
  \item Reinterpret the original hypothesis and experiment.
  \item Collect data according to the original protocol, \alert{unless flawed}. \only<article>{It is possible that the original experimental protocol had flaws. Then the new study should try and address this through an improved data collection process. For example, the original study might not have been double-blind. The new study can replicate the results in a double-blind regime.}
  \item Run the analysis again, \alert{unless flawed}. \only<article>{It is possible that the original analysis had flaws. For example, possible correlations may not have been taken into account.}
  \item See if the conclusions are in agreement.
  \end{enumerate}
\end{block}
\end{frame}

\begin{frame}
  \frametitle{Learning outcomes}
  \begin{block}{Understanding}
    \begin{itemize}
    \item What is a hold-out set, cross-validation and bootstrapping.
    \item The idea of not reusing data input to an algorithm to evaluate it.
    \item The fact that algorithms can be implemented by both humans and machines.
    \end{itemize}
  \end{block}
  
  \begin{block}{Skills}
    \begin{itemize}
    \item Use git and notebooks to document your work.
    \item Use hold-out sets or cross-validation to compare parameters/algorithms in Python.
    \item Use bootstrapping to get estimates of uncertainty in Python.
    \end{itemize}
  \end{block}

  \begin{block}{Reflection}
    \begin{itemize}
    \item What is a good use case for cross-validation over hold-out sets?
    \item When is it a good idea to use bootstrapping?
    \item How can we use the above techniques to avoid the false discovery problem?
    \item Can these techniques fully replace independent replication?
    \end{itemize}
  \end{block}
  
\end{frame}


%%% Local Variables:
%%% mode: latex
%%% TeX-master: "notes"
%%% End:


\include{reproducibility-assignment}
\section{Beliefs and probabilities}
\only<presentation>{
  \begin{frame}
    \tableofcontents[ 
    currentsection, 
    hideothersubsections, 
    sectionstyle=show/shaded
    ] 
  \end{frame}
}


\only<article>{Probability can be used to describe purely chance events, as in for example quantum physics. However, it is mostly used to describe uncertain events, such as the outcome of a dice roll or a coin flip, which only appear random. In fact, one can take it even further than that, and use it to model subjective uncertainty about any arbitrary event. Although probabilities are not the only way in which we can quantify uncertainty, it is a simple enough model, and with a rich enough history in mathematics, statistics, computer science and engineering that it is the most useful.}
\begin{frame}
  \frametitle{Uncertainty}
  \begin{itemize}
  \item We cannot perfectly predict the future.
  \item We cannot know for sure what happened in the past.
  \item How can we quantify this uncertainty?
  \item Probabilities!
  \end{itemize}
  \begin{block}{Axioms of probability}
    For any probability measure\footnote{$\Sigma$ is the set of possible events, with $A \in \Sigma$ always $A \subset \Omega$. Technically $\Sigma$ is a $\sigma$-algebra} $P$ on $(\Omega, \Sigma)$,
    \begin{enumerate}
    \item<2-> The probability of the certain event is $P(\Omega) = 1$
    \item<3->The probability of the impossible event is
      $P(\emptyset) = 0$
    \item<4->The probability of any event $A \in \Sigma$ is $0 \leq P(A) \leq 1$.
    \item<5-> If $A, B$ are disjoint, i.e. $A \cap B = \emptyset$, meaning
      that they cannot happen at the same time, then
      \[
      P(A \cup B) = P(A) + P(B)
      \]
    \end{enumerate}
  \end{block}
\end{frame}

\begin{frame}
  \only<article>{ Sometimes we would like to calculate the probability
    of some event $A$ happening given that we know that some other
    event $B$ has happened. For this we need to first define the idea
    of conditional probability.  }
  \begin{definition}[Conditional probability]
    The probability of $A$ happening if we know that $B$ has happened
    is defined to be:
    \[
    P(A \mid B) \defn \frac{P(A \cap B) }{P(B)}.
  \]
\end{definition}
\only<1>{
  Conditional probabilities obey the same rules as probabilities. }
  \only<article>{
    Here, the probability measure of any event $A$ given $B$ is defined to be the probability of the intersection of of the events divided by the second event.
    We can rewrite this definition as follows, by using the definition for $P(B \mid A)$}
  \begin{block}{Bayes's theorem}
    For $P(A_1 \cup A_2)  = 1$, $A_1 \cap A_2 = \emptyset$,
    \[
      P(A_i \mid B)
      \uncover<2->{= \frac{P(B \mid A_i) P(A_i)}{P(B)}}
      \uncover<3->{= \frac{P(B \mid A_i) P(A_i)}{P(B \mid A_1) P(A_1) + P(B \mid A_2) P(A_2)}}
    \]
  \end{block}
  \uncover<4->{
  \begin{example}[probability of rain]
    What is the probability of rain given a forecast $x_1$ or $x_2$?
    \begin{columns}
      \begin{column}{0.33\textwidth}
        \begin{table}[H]
          \centering
          \begin{tabular}{c|c}
            $\outcome_1$: rain & $P(\outcome_1) = 80\%$ \\
            $\outcome_2$: dry & $P(\outcome_2) = 20\%$
          \end{tabular}
          \caption{Prior probability of rain tomorrow}
        \end{table}
      \end{column}
      
      \begin{column}{0.33\textwidth}
        \uncover<5->{
        \begin{table}[H]
          \centering
          \begin{tabular}{c|c}
            $x_1$: rain & $P(x_1 \mid \outcome_1) = 90\%$ \\
            $x_2$: dry & $P(x_2 \mid \outcome_2) = 50\%$
          \end{tabular}
          \caption{Probability the forecast is correct}
        \end{table}
        }
      \end{column}
      \uncover<6->{
      \begin{column}{0.33\textwidth}
        \begin{table}[H]
          \centering
          \begin{tabular}{c}
            $P(\outcome_1 \mid x_1) = 87.8\%$ \\
            $P(\outcome_1 \mid x_2) = 44.4\%$
          \end{tabular}
          \caption{Probability that it will rain given the forecast}
        \end{table}
        }
      \end{column}
    \end{columns}
  \end{example}
  }
\end{frame}


\begin{frame}
  \frametitle{Classification in terms of conditional probabilities}
  \only<presentation>{
    \begin{itemize}
    \item Features $x_t \in \CX$.
    \item Class label $y_t \in \CY$.
    \item Probability model $P_\model(x_t \mid y_t)$.
    \item Prior class probability $P_\model(y_t = c)$.
    \end{itemize}
  }
  \only<article>{
    Conditional probability naturally appears in classification problems. Given a new example vector of data $x_t \in \CX$, we would like to calculate the probability of different classes $c \in \CY$ given the data, $P_\model(y_t = c \mid x_t)$.  
If we somehow obtained the distribution of data $P_\model(x_t \mid y_t)$ for each possible class, as well as the prior class probability $P_\model(y_t = c)$, 
from Bayes's theorem, we see that we can obtain the probability of the class:
  }
  \[
  P_\model(y_t = c \mid x_t) = \frac{P_\model(x_t \mid y_t = c) P_\model(y_t = c)}{\sum_{c' \in \CY} P_\model(x_t \mid y_t = c') P_\model(y_t = c')}
  \]
  \only<article>{
    for any class $c$. This directly gives us a method for classifying new data, as long as we have a way to obtain $P_\model(x_t \mid y_t)$ and $P_\model(y_t)$.
  }
  \only<1>{
        \begin{tikzpicture}
          \node[RV] at (0,0) (x) {$y_t$};
          \node[RV] at (0,2) (y) {$x_t$};
          \node[RV] at (1,1) (m) {$\model$};
          \draw[->] (x) to (y);
          \draw[->] (m) to (x);
          \draw[->] (m) to (y);
        \end{tikzpicture}
      }
  \uncover<2->{
    \begin{example}[Normal distribution]
 \only<2,4>{\includegraphics[width=0.5\textwidth]{../figures/equal-variance}Equal prior and variance}
      \only<3>{\includegraphics[width=0.5\textwidth]{../figures/unequal-variance}Unequal variance}
\only<5>{\includegraphics[width=0.5\textwidth]{../figures/unequal-prior}Unequal prior}
    \end{example}
  }
  \uncover<5>{
    \alert{But how can we get a probability model in the first place?}
  }
\end{frame}


  \begin{frame}
    \frametitle{Subjective probability}
    \only<article>{While probabilities apply to truly random events, they are also useful for representing subjective uncertainty. In this course, we will use a special symbol for subjective probability, $\bel$.}
    \begin{block}{Subjective probability measure $\bel$}
      \begin{itemize}
      \item If we think event $A$ is more likely than $B$, then $\bel(A) > \bel(B)$.
      \item Usual rules of probability apply:
        \begin{enumerate}
        \item $\bel(A) \in [0,1]$.
        \item $\bel(\emptyset) = 0$.
        \item If $A \cap B = \emptyset$, then $\bel(A \cup B) = \bel(A) + \bel(B)$.
        \end{enumerate}
      \end{itemize}
    \end{block}
  \end{frame}


  \begin{frame}
    \frametitle{Bayesian inference illustration}
    \begin{columns}
      \begin{column}{0.7\textwidth}
        \begin{block}{Use a subjective belief $\bel(\model)$ on $\Model$}
          \begin{itemize}
          \item<1-> \alert{Prior} belief $\bel(\model)$ represents our initial uncertainty.
          \item<2-> We \alert{observe history} $h$.
          \item<3->Each possible $\model$ assigns a \alert{probability} $P_\model(h)$ to $h$.
          \item<4-> We can use this to \alert{update} our belief via Bayes' theorem to obtain the \alert{posterior} belief:
            \[
            \bel(\model \mid h) \propto P_\model(h) \bel(\model)
            \tag{conclusion = evidence $\times$ prior}
            \]
          \end{itemize}
        \end{block}
      \end{column}
      \begin{column}{0.3\textwidth}
        \centering
\uncover<1->{\includegraphics[width=0.5\fwidth]{../figures/rl_worlds}
          \\
          prior
        }
        \\
        \uncover<2->{\includegraphics[width=0.5\fwidth]{../figures/rl_observations}
          \\
          evidence
        }
        \\
        \uncover<4->{\includegraphics[width=0.5\fwidth]{../figures/rl_worlds2}
          \\ 
          conclusion
        }
      \end{column}
    \end{columns}
  \end{frame}




  \subsection{Probability and Bayesian inference}
  \only<article>{One of the most important methods in machine learning
    and statistics is that of Bayesian inference.  This is the most
    fundamental method of drawing conclusions from data and explicit
    prior assumptions. In Bayesian inference, prior assumptions are
    represented as a probabilities on a space of hypotheses. Each
    hypothesis is seen as a probabilistic model of all possible data
    that we can see.}

  \only<article>{Frequently, we want to draw conclusions from data. However, the conclusions are never solely inferred from data, but also depend on prior assumptions about reality.}




  \begin{frame}
    \frametitle{Some examples}

    \begin{example}
      John claims to be a medium. He throws a coin $n$ times and predicts its value always correctly. Should we believe that he is a medium?
      \begin{itemize}
      \item $\model_1$: John is a medium.
      \item $\model_0$: John is not a medium.
      \end{itemize}
    \end{example}
    The answer depends on what we \alert{expect} a medium to be able to do, and how likely we thought he'd be a medium in the first place.

    \only<article>{
    \begin{example}
      Traces of DNA are found at a murder scene. We perform a DNA test against a database of $10^4$ citizens registered to be living in the area. We know that the probability of a false positive (that is, the test finding a match by mistake) is $10^{-6}$. If there is a match in the database, does that mean that the citizen was at the scene of the crime?
    \end{example}
    }
  \end{frame}





  \begin{frame}
    \frametitle{Bayesian inference}
    \only<article>{
      Now let us apply this idea to our specific problem. We already have the probability of the observation for each model, but we just need to define a \emph{prior probability} for each model. Since this is usually completely subjective, we give it another symbol.
    }
    \only<article>{
      \begin{block}{Prior probability}
        The prior probability $\bel$ on a set of models $\Model$ specifies our subjective belief $\bel(\model)$ that each model is true.\footnote{More generally $\bel$ is a probability measure.}
      \end{block}
    }
    \only<article>{
      This allows us to calculate the probability of John being a medium, given the data:
      \[
      \bel(\model_1 \mid \bx) = \frac{\Pr(\bx \mid \model_1) \bel(\model_1)}{\Pr_\bel(\bx)},
      \]
      where
      \[
      \Pr_\bel(\bx) \defn \Pr(\bx \mid \model_1) \bel(\model_1) + \Pr(\bx \mid \model_0) \bel(\model_0).
      \]
      The only thing left to specify is $\bel(\model_1)$, the probability that John is a medium before seeing the data. This is our subjective prior belief that mediums exist and that John is one of them.
      More generally, we can think of Bayesian inference as follows: }
    \begin{itemize}
    \item<1-> \only<article>{We start with a set of } mutually exclusive models $\Model = \{\model_1, \ldots, \model_k\}$.
    \item<2->\only<article>{Each model $\model$ is represented by a specific probabilistic model for any possible data $x$, that is}
      \only<presentation>{Probability model for any data $x$:} $P_\model(x) \equiv \Pr(x \mid \model)$.
    \item<3-> For each model, we have a prior probability $\bel(\model)$ that it is correct.
    \item<4-> \only<article>{After observing the data, we can calculate a posterior probability that the model is correct:}
      \only<presentation>{Posterior probability}
      \[
      \bel(\model \mid x) = \frac{\Pr(x \mid \model) \bel(\model)}{\sum_{\model' \in \Model} \Pr(x \mid \model') \bel(\model')}
      = \frac{P_\model(x) \bel(\model)}{\sum_{\model' \in \Model} P_{\model'} (x) \bel(\model')}.
      \]
    \end{itemize}
    \only<5->{
      \begin{block}{Interpretation}
        \begin{itemize}
        \item $\CM$: Set of all possible models that could describe the data.
        \item $P_\model(x)$: Probability of $x$ under model $\model$.
        \item Alternative notation $\Pr(x \mid \model)$: Probability of $x$ given that model $\model$ is correct.
        \item $\bel(\model)$: Our belief, before seeing the data, that $\model$ is correct.
        \item $\bel(\model \mid x)$: Our belief, aftering seeing the data, that $\model$ is correct.
      \end{itemize}
    \end{block}
    \only<article>{It must be emphasized that $P_\model(x) = \Pr(x \mid \model)$ as they are simply two different notations for the same thing. In words the first can be seen as the probability that model $\model$ assigns to data $x$, while the second as the probability of $x$ if $\model$ is the true model.}
  }
    \only<article>{
      Combining the prior belief with evidence is key in this procedure. Our posterior belief can then be used as a new prior belief when we get more evidence.}
  \end{frame}
\begin{frame}
\begin{exercise}[Continued example for medium]
    \only<article>{ Now let us apply this idea to our specific
      problem. We first make an independence assumption. In particular, we can assume that success and failure comes from a Bernoulli distribution with a parameter depending on the model.}
    \begin{align}
    P_{\model} (x) &= \prod_{t=1}^n P_{\model} (x_t).
\tag{independence property}
    \end{align}
    \only<article>{We first need to specify how well a medium could predict. Let's assume that a true medium would be able to predict perfectly, and that a non-medium would only predict randomly. This leads to the following models:}
    \begin{align}
      P_{\model_1}(x_t = 1) &= 1, &P_{\model_1}(x_t = 0) &= 0.
                                                           \tag{true medium model}
                                                           \\
      P_{\model_0}(x_t = 1) &= 1/2, &P_{\model_0}(x_t = 0) &= 1/2.
                                                             \tag{non-medium model}
    \end{align}
    \only<article>{
      The only thing left to specify is $\bel(\model_1)$, the probability
      that John is a medium before seeing the data. This is our
      subjective prior belief that mediums exist and that John is one of
      them.}
    \uncover<3->{
      \begin{align}
        \bel(\model_0) &= 1/2,   &  \bel(\model_1) &= 1/2.
                                                     \tag{prior belief}
      \end{align}
    }
    \only<article>{Combining the prior belief with evidence is key in this
      procedure. Our posterior belief can then be used as a new prior
      belief when we get more evidence.  }
    \uncover<4>{
      \begin{align}
      \bel(\model_1 \mid x) & = \frac{P_{\model_1}(x)
        \bel(\model_1)}{\Pr_\bel(x)} \tag{posterior belief}
        \\
      \Pr_\bel(x) &\defn P_{\model_1}(x) \bel(\model_1) + P_{\model_0}(x) \bel(\model_0).
\tag{marginal distribution}
      \end{align}
    }
    Throw a coin 4 times, and have a classmate make a prediction. What your belief that your classmate is a medium? Is the prior you used reasonable?
  \end{exercise}
  \end{frame}


\begin{frame}
  \frametitle{Sequential update of beliefs}
    \only<article>{Assume you have $n$ meteorologists. At each day $t$, each meteorologist $i$ gives a probability $p_{t,\model_i}\defn P_{\model_i}(x_t = \textrm{rain})$ for rain. Consider the case of there being three meteorologists, and each one making the following prediction for the coming week. Start with a uniform prior $\bel(\model) = 1/3$ for each model.}
    {
      \begin{table}[h]
        \begin{tabular}{c|l|l|l|l|l|l|l}
          &M&T&W&T&F&S&S\\
          \hline
          CNN & 0.5 & 0.6 & 0.7 & 0.9 & 0.5 & 0.3 & 0.1\\
          SMHI & 0.3 & 0.7 & 0.8 & 0.9 & 0.5 & 0.2 & 0.1\\
          YR & 0.6 & 0.9 & 0.8 & 0.5 & 0.4 & 0.1 & 0.1\\
          \hline
          Rain? & Y & Y & Y & N & Y & N & N
        \end{tabular}
        \caption{Predictions by three different entities for the probability of rain on a particular day, along with whether or not it actually rained.}
        \label{tab:meteorologists}
      \end{table}
    }
  \begin{exercise}
    \begin{itemize}
    \item $n$ meteorological stations $\cset{\mdp_i}{i=1, \ldots,n}$
    \item The $i$-th station predicts rain $P_{\mdp_i}(x_t \mid x_1, \ldots, x_{t-1})$.
    \item Let $\bel_t(\mdp)$ be our belief at time $t$.
      Derive the next-step belief
      $\bel_{t+1}(\mdp) \defn  \bel_t(\mdp | y_{t})$ in terms of the current belief $\bel_t$.
    \item Write a python function that computes this posterior
    \end{itemize}
  \end{exercise}
  \uncover<2->{
    \[
      \bel_{t+1}(\mdp)
      \defn
      \bel_t(\mdp | x_{t})
      =
      \frac{P_\mdp(x_t \mid x_1, \ldots, x_{t-1}) \bel_t(\mdp)}
      {\sum_{\mdp'} P_{\mdp'}(x_t \mid x_1, \ldots, x_{t-1}) \bel_t(\mdp')}
    \]
  }
\end{frame}



\begin{frame}[label=beta-example]
  \frametitle{Bayesian inference for Bernoulli distributions}
  \only<1>{
    \begin{block}{Estimating a coin's bias}
      A fair coin comes heads $50\%$ of the time. 
      We want to test an unknown coin, which we think may not be completely fair. 
    \end{block}
  }
  \only<1,2>{
    \begin{figure}[h]
      \centering
      \includegraphics[width=\textwidth]{../figures/beta-prior}
      \caption{Prior belief $\bel$ about the coin bias $\theta$.}
    \end{figure}
  }
  \only<2>{
    For a sequence of throws $x_t \in \{0,1\}$,
    \[
    P_\theta(x) \propto \prod_t \theta^{x_t} (1 - \theta)^{1 - x_t}
    = \theta^{\textrm{\#Heads}} (1 - \theta)^{\textrm{\#Tails}}
    \]
  }
  \only<3>{
    \begin{figure}[h]
      \centering
      \includegraphics[width=\textwidth]{../figures/beta-likelihood}
      \caption{Prior belief $\bel$ about the coin bias $\theta$ and likelihood of $\theta$ for the data.}
    \end{figure}
    Say we throw the coin 100 times and obtain 70 heads. Then we plot the \alert{likelihood} $P_\theta(x)$ of different models.
  }
  \only<4>{
    \begin{figure}[h]
      \centering
      \includegraphics[width=\textwidth]{../figures/beta-posterior}
      \caption{Prior belief $\bel(\theta)$ about the coin bias $\theta$, likelihood of $\theta$ for the data, and posterior belief $\bel(\theta \mid x)$}
    \end{figure}
    From these, we calculate a \alert{posterior} distribution over the correct models. This represents our conclusion given our prior and the data.
  }
  \only<article>{If the prior distribution is described by the so-called Beta density
    \[
    f(\theta \mid \alpha, \beta) \propto \theta^{\alpha -1} (1 - \theta)^{\beta -1}
    \]
    where $\alpha, \beta$ describe the shape of the Beta distribution.
  }
\end{frame}



\begin{frame}
  \frametitle{Learning outcomes}
  \begin{block}{Understanding}
    \begin{itemize}
    \item The axioms of probability, marginals and conditional distributions.
    \item The philosophical underpinnings of Bayesianism.
    \item The simple conjugate model for Bernoulli distributions.
    \end{itemize}
  \end{block}
  
  \begin{block}{Skills}
    \begin{itemize}
    \item Be able to calculate with probabilities using the marginal and conditional definitions and Bayes rule.
    \item Being able to implement a simple Bayesian inference algorithm in Python.
    \end{itemize}
  \end{block}

  \begin{block}{Reflection}
    \begin{itemize}
    \item How useful is the Bayesian representation of uncertainty?
    \item How restrictive is the need to select a prior distribution?
    \item Can you think of another way to explicitly represent uncertainty in a way that can incorporate new evidence?
    \end{itemize}
  \end{block}
  
\end{frame}

  %%% Local Variables:
  %%% mode: latex
  %%% TeX-engine: xetex
  %%% TeX-master: "notes.tex"
  %%% End:
 % Bayesian inference
\section{Hierarchies of decision making problems}
\only<presentation>{
  \begin{frame}
    \tableofcontents[ 
    currentsection, 
    hideothersubsections, 
    sectionstyle=show/shaded
    ] 
  \end{frame}
}


\only<article>{
  All machine learning problems are essentially decision problems. This essentially means replacing some human decisions with machine decisions. One of the simplest decision problems is classification, where you want an algorithm to decide the correct class of some data, but even within this simple framework there is a multitude of decisions to be made. The first is how to frame the classification problem the first place. The second is how to collect, process and annotate the data. The third is choosing the type of classification model to use. The fourth is how to use the collected data to find an optimal classifier within the selected type. After all this has been done, there is the problem of classifying new data. In this course, we will take a holistic view of the problem, and consider each problem in turn, starting from the lowest level and working our way up.}


\subsection{Simple decision problems}
\begin{frame}
  \frametitle{Preferences}
  \only<article>{The simplest decision problem involves selecting one item from a set of choices, such as in the following examples}  
  \begin{example}
    \begin{block}{Food}
      \begin{itemize}
      \item[A] McDonald's cheeseburger
      \item[B] Surstromming
      \item[C] Oatmeal
      \end{itemize}
    \end{block}
    \begin{block}{Money}
      \begin{itemize}
      \item[A] 10,000,000 SEK
      \item[B] 10,000,000 USD
      \item[C] 10,000,000 BTC
      \end{itemize}
    \end{block}
    \begin{block}{Entertainment}
      \begin{itemize}
      \item[A] Ticket to Liseberg
      \item[B] Ticket to Rebstar
      \item[C] Ticket to Nutcracker
      \end{itemize}
    \end{block}
  \end{example}
\end{frame}

\begin{frame}
  \frametitle{Rewards and utilities}
  \only<article>{In the decision theoretic framework, the things we receive are called rewards, and we assign a utility value to each one of them, showing which one we prefer.}
  \begin{itemize}
  \item Each choice is called a \alert{reward} $r \in \CR$.
  \item There is a \alert{utility function} $U : \CR \to \Reals$, assigning values to reward.
  \item We (weakly) prefer $A$ to $B$ iff $U(A) \geq U(B)$.
  \end{itemize}
  \only<article>{In each case, given $U$ the choice between each reward is trivial. We just select the reward:
    \[
    r^* \in \argmax_r U(r)
    \]
    The main difficult is actually selecting the appropriate utility function. In a behavioural context, we simply assume that humans act with respect to a specific utility function. However, figuring out this function from behavioural data is non trivial. ven when this assumption is correct, individuals do not have a common utility function.
  }
  \begin{exercise}
    From your individual preferences, derive a \alert{common utility function} that reflects everybody's preferences in the class for each of the three examples. Is there a simple algorithm for deciding this? Would you consider the outcome fair?
  \end{exercise}
\end{frame}

\begin{frame}
  \frametitle{Preferences among random outcomes}
  \begin{example}
    Would you rather \ldots
    \begin{itemize}
    \item[A] Have 100 EUR now?
    \item[B] Flip a coin, and get 200 EUR if it comes heads?
    \end{itemize}    
  \end{example}
  \uncover<2->{
    \begin{block}{The expected utility hypothesis}
      Rational decision makers prefer choice $A$ to $B$ if
      \[
      \E(U | A) \geq \E(U | B),
      \]
      where the expected utility is
      \[
      \E(U | A) = \sum_r U(r) \Pr(r | A).
      \]
    \end{block}
    In the above example, $r \in \{0, 100, 200\}$ and $U(r)$ is
    increasing, and the coin is fair.
  }
  \begin{itemize}
  \item<3-> If $U$ is convex, we prefer B.
  \item<4-> If $U$ is concave, we prefer A.
  \item<5-> If $U$ is linear, we don't care.
  \end{itemize}
\end{frame}


\begin{frame}
  \frametitle{Uncertain rewards}
  \only<article>{However, in real life, there are many cases where we can only choose between uncertain outcomes. The simplest example are lottery tickets, where rewards are essentially random. However, in many cases the rewards are not really random, but simply uncertain. In those cases it is useful to represent our uncertainty with probabilities as well, even though there is nothing really random.}
  \begin{itemize}
  \item Decisions $\decision \in \Decision$
  \item Each choice is called a \alert{reward} $r \in \CR$.
  \item There is a \alert{utility function} $U : \CR \to \Reals$, assigning values to reward.
  \item We (weakly) prefer $A$ to $B$ iff $U(A) \geq U(B)$.
  \end{itemize}

  \begin{example}
    \begin{columns}
      \begin{column}{0.5\textwidth}
        You are going to work, and it might rain.  What do you do?
        \begin{itemize}
        \item $\decision_1$: Take the umbrella.
        \item $\decision_2$: Risk it!
        \item $\outcome_1$: rain
        \item $\outcome_2$: dry
        \end{itemize}
      \end{column}
      \begin{column}{0.5\textwidth}
        \begin{table}
          \centering
          \begin{tabular}{c|c|c}
            $\Rew(\outcome,\decision)$ & $\decision_1$ & $\decision_2$ \\ %ro: U has only one argument.
            \hline
            $\outcome_1$ & dry, carrying umbrella & wet\\
            $\outcome_2$ & dry, carrying umbrella & dry\\
            \hline
            \hline
            $U[\Rew(\outcome,\decision)]$ & $\decision_1$ & $\decision_2$ \\
            \hline
            $\outcome_1$ & 0 & -10\\
            $\outcome_2$ & 0 & 1
          \end{tabular}
          \caption{Rewards and utilities.}
          \label{tab:rain-utility-function}
        \end{table}

        \begin{itemize}
        \item<2-> $\max_\decision \min_\outcome U = 0$
        \item<3-> $\min_\outcome \max_\decision U = 0$
        \end{itemize}
      \end{column}

    \end{columns}
  \end{example}
\end{frame}



\begin{frame}
  \frametitle{Expected utility}
  \[
  \E (U \mid a) = \sum_r U[\Rew(\outcome, \decision)] \Pr(\outcome \mid \decision)
  \]
  \begin{example}%ro: rather an exercise?
    You are going to work, and it might rain. The forecast said that
    the probability of rain $(\outcome_1)$ was $20\%$. What do you do?
    \begin{itemize}
    \item $\decision_1$: Take the umbrella.
    \item $\decision_2$: Risk it!
    \end{itemize}
    \begin{table}
      \centering
      \begin{tabular}{c|c|c}
        $\Rew(\outcome,\decision)$ & $\decision_1$ & $\decision_2$ \\ %ro: U has only one argument.
        \hline
        $\outcome_1$ & dry, carrying umbrella & wet\\
        $\outcome_2$ & dry, carrying umbrella & dry\\
        \hline
        \hline
        $U[\Rew(\outcome,\decision)]$ & $\decision_1$ & $\decision_2$ \\
        \hline
        $\outcome_1$ & 0 & -10\\
        $\outcome_2$ & 0 & 1\\
        \hline
        \hline
        $\E_P(U \mid \decision)$ & 0 &  -1.2 \\ 
      \end{tabular}
      \caption{Rewards, utilities, expected utility for $20\%$ probability of rain.}
      \label{tab:rain-utility-function}
    \end{table}
  \end{example}
\end{frame}





\subsection{Decision rules}

\only<article>{We now move from simple decisions to decisions that
  depend on some observation. We shall start with a simple problem in applied meteorology. Then we will discuss hypothesis testing as a decision making problem. Finally, we will go through an exercise in Bayesian methods for classification.}

\begin{frame}
  \frametitle{Bayes decision rules}
  Consider the case where outcomes are independent of decisions:
  \[
  \util (\bel, \decision) \defn \sum_{\model}  \util (\model, \decision) \bel(\model)
  \]
  This corresponds e.g. to the case where $\bel(\model)$ is the belief about an unknown world.
  \begin{definition}[Bayes utility]
    \label{def:bayes-utility}
    The maximising decision for $\bel$ has an expected utility equal to:
    \begin{equation}
      \BUtil(P) \defn \max_{\decision \in \Decision} \util (\bel, \decision).
      \label{eq:bayes-utility}
    \end{equation}
  \end{definition}
\end{frame}




\begin{frame}
  \frametitle{The $n$-meteorologists problem}
  \only<article>{Of course, we may not always just be interested in classification performance in terms of predicting the most likely class. It strongly depends on the problem we are actually wanting to solve. In  biometric authentication, for example, we want to guard against the unlikely event that an impostor will successfully be authenticated. Even if the decision rule that always says 'OK' has the lowest classification error in practice, the expected cost of impostors means that the optimal decision rule must sometimes say 'Failed' even if this leads to false rejections sometimes.}
  \begin{exercise}
    \only<presentation>{
      \only<1>{
        \begin{itemize}
        \item Meteorological models $\CM = \set{\model_1, \ldots, \model_n}$
        \item Rain predictions at time $t$: $p_{t,\model} \defn  P_{\model}(x_t = \textrm{rain})$.
        \item Prior probability $\bel(\model) = 1/n$ for each model.
        \item Should we take the umbrella?
        \end{itemize}
      }
    }
    \only<article>{Assume you have $n$ meteorologists. At each day $t$, each meteorologist $i$ gives a probability $p_{t,\model_i}\defn P_{\model_i}(x_t = \textrm{rain})$ for rain. Consider the case of there being three meteorologists, and each one making the following prediction for the coming week. Start with a uniform prior $\bel(\model) = 1/3$ for each model.}
    {
      \begin{table}[h]
        \begin{tabular}{c|l|l|l|l|l|l|l}
          &M&T&W&T&F&S&S\\
          \hline
          CNN & 0.5 & 0.6 & 0.7 & 0.9 & 0.5 & 0.3 & 0.1\\
          SMHI & 0.3 & 0.7 & 0.8 & 0.9 & 0.5 & 0.2 & 0.1\\
          YR & 0.6 & 0.9 & 0.8 & 0.5 & 0.4 & 0.1 & 0.1\\
          \hline
          Rain? & Y & Y & Y & N & Y & N & N
        \end{tabular}
        \caption{Predictions by three different entities for the probability of rain on a particular day, along with whether or not it actually rained.}
        \label{tab:meteorologists}
      \end{table}
    }
    \uncover<2->{
      \begin{enumerate}
      \item<2-> What is your belief about the quality of each meteorologist after each day? 
      \item<3-> What is your belief about the probability of rain each day? 
        \[
        P_\bel(x_t = \textrm{rain} \mid x_1, x_2, \ldots x_{t-1})
        =
        \sum_{\model \in \Model} P_\model(x_t = \textrm{rain} \mid x_1, x_2, \ldots x_{t-1})
        \bel(\model \mid x_1, x_2, \ldots x_{t-1}) 
        \]
      \item<4-> Assume you can decide whether or not to go running each
        day. If you go running and it does not rain, your utility is 1. If
        it rains, it's -10. If you don't go running, your utility is
        0. What is the decision maximising utility in expectation (with respect to the posterior) each
        day?
      \end{enumerate}
    }
  \end{exercise}
\end{frame}


\subsection{Statistical testing}
\only<article>{A common type of decision problem is a statistical test. This arises when we have a set of possible candidate models $\CM$ and we need to be able to decide which model to select after we see the evidence.
  Many times, there is only one model under consideration, $\model_0$, the so-called \alert{null hypothesis}. Then, our only decision is whether or not to accept or reject this hypothesis.}
\begin{frame}
  \frametitle{Simple hypothesis testing}
  \only<article>{Let us start with the simple case of needing to compare two models.}
  \begin{block}{The simple hypothesis test as a decision problem}
    \begin{itemize}
    \item $\CM = \{\model_0, \model_1\}$
    \item $a_0$: Accept model $\model_0$
    \item $a_1$: Accept model $\model_1$
    \end{itemize}
    \begin{table}[H]
      \begin{tabular}{c|cc}
        $\util$& $\model_0$& $\model_1$\\\hline
        $a_0$ & 1 & 0\\
        $a_1$ & 0 & 1
      \end{tabular}
      \caption{Example utility function for simple hypothesis tests.}
    \end{table}
    \only<article>{There is no reason for us to be restricted to this utility function. As it is diagonal, it effectively treats both types of errors in the same way.}
  \end{block}

  \begin{example}[Continuation of the medium example]
    \begin{itemize}
    \item $\model_1$: that John is a medium.
    \item $\model_0$: that John is not a medium.
    \end{itemize}
    \only<article>{
      Let $x_t$ be $0$ if John makes an incorrect prediction at time $t$ and $x_t = 1$ if he makes a correct prediction. Let us once more assume a Bernoulli model, so that John's claim that he can predict our tosses perfectly means that for a sequence of tosses $\bx = x_1, \ldots, x_n$,
      \[
      P_{\model_1}(\bx) = \begin{cases}
        1, & x_t = 1 \forall t \in [n]\\
        0, & \exists t \in [n] : x_t = 0.
      \end{cases}
      \]
      That is, the probability of perfectly correct predictions is 1, and that of one or more incorrect prediction is 0. For the other model, we can assume that all draws are independently and identically distributed from a fair coin. Consequently, no matter what John's predictions are, we have that:
      \[
      P_{\model_0}(\bx = 1 \ldots 1) = 2^{-n}.
      \]
      So, for the given example, as stated, we have the following facts:
      \begin{itemize}
      \item If John makes one or more mistakes, then $\Pr(\bx \mid \model_1) = 0$ and $\Pr(\bx \mid \model_0) = 2^{-n}$. Thus, we should perhaps say that then John is not a medium
      \item If John makes no mistakes at all, then 
        \begin{align}
          \Pr(\bx = 1, \ldots, 1 \mid \model_1) &= 1,
          &
            \Pr(\bx = 1, \ldots, 1 \mid \model_0) &= 2^{-n}.
        \end{align}
      \end{itemize}
      Now we can calculate the posterior distribution, which is
      \[
      \bel(\model_1 \mid \bx = 1, \ldots, 1) = \frac{1 \times \bel(\model_1)}{1 \times \bel(model_1) + 2^{-n} (1 - \bel(\model_1))}.
      \]
      Our expected utility for taking action $a_0$ is actually
    }
    \[
    \E_\bel(\util \mid a_0) = 1 \times \bel(\model_0 \mid \bx) + 0 \times \bel(\model_1 \mid \bx), \qquad
    \E_\bel(\util \mid a_1) = 0 \times \bel(\model_0 \mid \bx) + 1 \times \bel(\model_1 \mid \bx)
    \]
  \end{example}
  
\end{frame}


\begin{frame}
  \frametitle{Null hypothesis test}
  Many times, there is only one model under consideration, $\model_0$, the so-called \alert{null hypothesis}. \only<article>{ This happens when, for example, we have no simple way of defining an appropriate alternative. Consider the example of the medium: How should we expect a medium to predict? Then, our only decision is whether or not to accept or reject this hypothesis.}
  \begin{block}{The null hypothesis test as a decision problem}
    \begin{itemize}
    \item $a_0$: Accept model $\model_0$
    \item $a_1$: Reject model $\model_0$
    \end{itemize}
  \end{block}

  \begin{example}{Construction of the test for the medium}
    \begin{itemize}
    \item<2-> $\model_0$ is simply the $\Bernoulli(1/2)$ model: responses are by chance.
    \item<3-> We need to design a policy $\pol(a \mid \bx)$ that accepts or rejects depending on the data.
    \item<4-> Since there is no alternative model, we can only construct this policy according to its properties when $\model_0$ is true.
    \item<5-> In particular, we can fix a policy that only chooses $a_1$ when $\model_0$ is true a proportion $\delta$ of the time.
    \item<6-> This can be done by construcing a threshold test from the inverse-CDF.
    \end{itemize}
  \end{example}
\end{frame}
\begin{frame}
  \frametitle{Using $p$-values to construct statistical tests}
  \begin{definition}[Null statistical test]
    \only<article>{
      A statistical test $\pol$ is a decision rule for accepting or rejecting a hypothesis on the basis of evidence. A $p$-value test rejects a hypothesis whenever the value of the statistic $f(x)$ is smaller than a threshold.}
    The statistic $f : \CX \to [0,1]$ is  designed to have the property:
    \[
    P_{\model_0}(\cset{x}{f(x) \leq \delta}) = \delta.
    \]
    If our decision rule is:
    \[
    \pol(a \mid x) =
    \begin{cases}
      a_0, & f(x) \leq \delta\\
      a_1, & f(x) > \delta,
    \end{cases}
    \]
    the probability of rejecting the null hypothesis when it is true is exactly $\delta$.
  \end{definition}
  \only<presentation>{The value of the statistic $f(x)$, otherwise known as the \alert{$p$-value}, is uninformative.}
  \only<article>{This is because, by definition, $f(x)$ has a uniform distribution under $\model_0$. Hence the value of $f(x)$ itself is uninformative: high and low values are equally likely. In theory we should simply choose $\delta$ before seeing the data and just accept or reject based on whether $f(x) \leq \delta$. However nobody does that in practice, meaning that $p$-values are used incorrectly. Better not to use them at all, if uncertain about their meaning.}
\end{frame}
\begin{frame}
  \frametitle{Issues with $p$-values}
  \begin{itemize}
  \item They only measure quality of fit \alert{on the data}.
  \item Not robust to model misspecification. \only<article>{For example, zero-mean testing using the $\chi^2$-test has a normality assumption.}
  \item They ignore effect sizes. \only<article>{For example, a linear analysis may determine that there is a significant deviation from zero-mean, but with only a small effect size of 0.01. Thus, reporting only the $p$-value is misleading}
  \item They do not consider prior information. 
  \item They do not represent the probability of having made an error. \only<article>{In particular, a $p$-value of $\delta$ does not mean that the probability that the null hypothesis is false given the data $x$, is $\delta$, i.e. $\delta \neq \Pr(\neg \model_0 \mid x)$.}
  \item The null-rejection error probability is the same irrespective of the amount of data (by design).
  \end{itemize}
\end{frame}

\begin{frame}\frametitle{$p$-values for the medium example}
  \only<article>{Let us consider the example of the medium.}
  \begin{itemize}
  \item<2->$\model_0$ is simply the $\Bernoulli(1/2)$ model:
    responses are by chance. 
  \item<3->CDF: $P_{\model_0}(N \leq n \mid K = 100)$ \only<article> {is the probability of at most $N$ successes if we throw the coin 100 times. This is in fact the cumulative probability function of the binomial distribution. Recall that the binomial represents the distribution for the number of successes of independent experiments, each following a Bernoulli distribution.}
  \item<4->ICDF:  the number of successes that will happen with probability at least $\delta$
  \item<5->e.g. we'll get at most 50 successes a proportion $\delta = 1/2$ of the time.
  \item<6>Using the (inverse) CDF we can construct a policy $\pol$ that selects $a_1$ when $\model_0$ is true only a $\delta$ portion of the time, for any choice of $\delta$.
  \end{itemize}
  \begin{columns}
    \setlength\fheight{0.33\columnwidth}
    \setlength\fwidth{0.33\columnwidth}
    \begin{column}{0.5\textwidth}
      \only<3,4,5,6>{\input{../figures/binomial-cdf.tikz}}      
    \end{column}
    \begin{column}{0.5\textwidth}
      \only<4,5,6>{\input{../figures/binomial-icdf.tikz}}
    \end{column}
  \end{columns}    
\end{frame}



\begin{frame}
  \frametitle{Building a test}
  \begin{block}{The test statistic}
    We want the test to reflect that we don't have a significant number of failures.
    \[
    f(x) = 1 - \textrm{binocdf}(\sum_{t=1}^n x_t, n, 0.5)
    \]
  \end{block}
  \begin{alertblock}{What $f(x)$ is and is not}
    \begin{itemize}
    \item It is a \textbf{statistic} which is $\leq \delta$ a $\delta$ portion of the time when $\model_0$ is true.
    \item It is \textbf{not} the probability of observing $x$ under $\model_0$.
    \item It is \textbf{not} the probability of $\model_0$ given $x$.
    \end{itemize}
  \end{alertblock}
\end{frame}
\begin{frame}
  \begin{exercise}
    \begin{itemize}
    \item<1-> Let us throw a coin 8 times, and try and predict the outcome.
    \item<2-> Select a $p$-value threshold so that $\delta = 0.05$. 
      For 8 throws, this corresponds to \uncover<3->{$ > 6$ successes or $\geq 87.5\%$ success rate}.
    \item<3-> Let's calculate the $p$-value for each one of you
    \item<4-> What is the rejection performance of the test?
    \end{itemize}
    \setlength\fheight{0.25\columnwidth}
    \setlength\fwidth{0.5\columnwidth}
    \only<2,3>{
      \begin{figure}[H]
        \input{../figures/p-value-example-rejection-threshold.tikz}
        \caption{Here we see how the rejection threshold, in terms of the success rate, changes with the number of throws to achieve an error rate of $\delta = 0.05$.}
      \end{figure}
      \only<article>{As the amount of throws goes to infinity, the threshold converges to $0.5$. This means that a statistically significant difference from the null hypothesis can be obtained, even when the actual model from which the data is drawn is only slightly different from 0.5.}
    }
    \only<4>{
      \begin{figure}[H]
        \input{../figures/p-value-example-rejection.tikz}
        \caption{Here we see the rejection rate of the null hypothesis ($\model_0$) for two cases. Firstly, for the case when $\model_0$ is true. Secondly, when the data is generated from $\Bernoulli(0.55)$.}
      \end{figure}
      \only<article>{As we see, this method keeps its promise: the null is only rejected 0.05 of the time when it's true. We can also examine how often the null is rejected when it is false... but what should we compare against? Here we are generating data from a $\Bernoulli(0.55)$ model, and we can see the rejection of the null increases with the amount of data. This is called the \alert{power} of the test with respect to the $\Bernoulli(0.55)$ distribution. }
    }
  \end{exercise}
\end{frame}

\begin{frame}
  \begin{alertblock}{Statistical power and false discovery.}
    Beyond not rejecting the null when it's true, we also want:
    \begin{itemize}
    \item High power: Rejecting the null when it is false.
    \item Low false discovery rate: Accepting the null when it is true.
    \end{itemize}
  \end{alertblock}
  \begin{block}{Power}
    The power depends on what hypothesis we use as an alternative.
    \only<article>{This implies that we cannot simply consider a plain null hypothesis test, but must formulate a specific alternative hypothesis. }
  \end{block}

  \begin{block}{False discovery rate}
    False discovery depends on how likely it is \alert{a priori} that the null is false.
    \only<article>{This implies that we need to consider a prior probability for the null hypothesis being true.}
  \end{block}

  \only<article>{Both of these problems suggest that a Bayesian approach might be more suitable. Firstly, it allows us to consider an infinite number of possible alternative models as the alternative hypothesis, through Bayesian model averaging. Secondly, it allows us to specify prior probabilities for each alternative. This is especially important when we consider some effects unlikely.}
\end{frame}

\begin{frame}
  \frametitle{The Bayesian version of the test}
  \begin{enumerate}
  \item Set $\util(a_i, \model_j) = \ind{i = j}$. \only<article>{This choice makes sense if we care equally about either type of error.}
  \item Set $\bel(\model_i) = 1/2$. \only<article>{Here we place an equal probability in both models.}
  \item $\model_0$: $\Bernoulli(1/2)$. \only<article>{This is the same as the null hypothesis test.}
  \item $\model_1$: $\Bernoulli(\theta)$, $\theta \sim \Uniform([0,1])$. \only<article>{This is an extension of the simple hypothesis test, with an alternative hypothesis that says ``the data comes from an arbitrary  Bernoulli model''.}
  \item Calculate $\bel(\model \mid x)$.
  \item Choose $a_i$, where $i = \argmax_{j} \bel(\model_j \mid x)$.
  \end{enumerate}

  \begin{block}{Bayesian model averaging for the alternative model $\model_1$}
    \only<article>{In this scenario, $\model_0$ is a simple point model, e.g. corresponding to a $\Bernoulli(1/2)$. However $\model_1$ is a marginal distribution integrated over many models, e.g. a $Beta$ distribution over Bernoulli parameters.}
    \begin{align}
      P_{\model_1}(x) &= \int_\Param B_{\param}(x) \dd \beta(\param) \\
      \bel(\model_0 \mid x) &= \frac{P_{\model_0}(x) \bel(\model_0)}
                              {P_{\model_0}(x) \bel(\model_0) + P_{\model_1}(x) \bel(\model_1)}
    \end{align}
  \end{block}
\end{frame}
\begin{frame}
  \only<1>{
    \begin{figure}[H]
      \input{../figures/p-value-example-posterior.tikz}
      \caption{Here we see the convergence of the posterior probability.}
    \end{figure}
    \only<article>{As can be seen in the figure above, in both cases, the posterior converges to the correct value, so it can be used to indicate our confidence that the null is true.}
  }
  \only<2>{
    \begin{figure}[H]
      \input{../figures/p-value-example-null-posterior.tikz}
      \caption{Comparison of the rejection probability for the null and the Bayesian test when $\model_0$ is true.}
    \end{figure}
    \only<article>{Now we can use this Bayesian test, with uniform prior, to see how well it performs. While the plain null hypothesis test has a fixed rejection rate of $0.05$, the Bayesian test's rejection rate converges to 0 as we collect more data.}
  }
  \only<3>{
    \begin{figure}[H]
      \input{../figures/p-value-example-true-posterior.tikz}
      \caption{Comparison of the rejection probability for the null and the Bayesian test when $\model_1$ is true.}
    \end{figure}
    \only<article>{However, both methods are able to reject the null hypothesis more often when it is false, as long as we have more data.}
  }
\end{frame}
\begin{frame}
  \frametitle{Further reading}
  \begin{block}{Points of significance (Nature Methods)}
    \begin{itemize}
    \item Importance of being uncertain \url{https://www.nature.com/articles/nmeth.2613}
    \item Error bars \url{https://www.nature.com/articles/nmeth.2659}
    \item P values and the search for significance \url{https://www.nature.com/articles/nmeth.4120}
    \item Bayes' theorem \url{https://www.nature.com/articles/nmeth.3335}
    \item Sampling distributions and the bootstrap \url{https://www.nature.com/articles/nmeth.3414}
    \end{itemize}
  \end{block}
\end{frame}


\section{Formalising Classification problems}
\only<article>{
  One of the simplest decision problems is classification. At the simplest level, this is the problem of observing some data point $x_t \in \CX$ and making a decision about what class $\CY$ it belongs to. Typically, a fixed classifier is defined as a decision rule $\pi(a | x)$ making decisions $a \in \CA$, where the decision space includes the class labels, so that if we observe some point $x_t$ and choose $a_t = 1$, we essentially declare that $y_t = 1$.

  Typically, we wish to have a classification policy that minimises classification error.
}
\begin{frame}
  \frametitle{Deciding a class given a model}
  \only<article>{In the simplest classification problem, we observe some features $x_t$ and want to make a guess $\decision_t$ about the true class label $y_t$. Assuming we have some probabilistic model $P_\model(y_t \mid x_t)$, we want to define a decision rule $\pol(\decision_t \mid x_t)$ that is optimal, in the sense that it maximises expected utility for $P_\model$.}
  \begin{itemize}
  \item Features $x_t \in \CX$.
  \item Label $y_t \in \CY$.
  \item Decisions $\decision_t \in \CA$.
  \item Decision rule $\pol(\decision_t \mid x_t)$ assigns probabilities to actions.
  \end{itemize}
  
  \begin{block}{Standard classification problem}
    \only<article>{In the simplest case, the set of decisions we make are the same as the set of classes}
    \[
    \CA = \CY, \qquad
    U(\decision, y) = \ind{\decision = y}
    \]
  \end{block}

  \begin{exercise}
    If we have a model $P_\model(y_t \mid x_t)$, and a suitable $U$, what is the optimal decision to make?
  \end{exercise}
  \only<presentation>{
    \uncover<2->{
      \[
      \decision_t \in \argmax_{\decision \in \Decision} \sum_y P_\model(y_t = y \mid x_t) \util(\decision, y)
      \]
    }
    \uncover<3>{
      For standard classification,
      \[
      \decision_t \in \argmax_{\decision \in \Decision} P_\model(y_t = \decision \mid x_t)
      \]
    }
  }
\end{frame}


\begin{frame}
  \frametitle{Deciding the class given a model family}
  \begin{itemize}
  \item Training data $\Training = \cset{(x_i, y_i)}{i=1, \ldots, \ndata}$
  \item Models $\cset{P_\model}{\model \in \Model}$.
  \item Prior $\bel$ on $\Model$.
  \end{itemize}
  \only<article>{Similarly to our example with the meteorological stations, we can define a posterior distribution over models.}
  \begin{block}{Posterior over classification models}
    \[
    \bel(\model \mid \Training) = \frac{P_\model(y_1, \ldots, y_\ndata \mid
      x_1, \ldots, x_\ndata) \bel(\model)} {\sum_{\model' \in \Model}
      P_{\model'}(y_1, \ldots, y_\ndata \mid x_1, \ldots, x_\ndata)
      \bel(\model')}
    \]
    \only<article>{
      This posterior form can be seen as weighing each model according to how well they can predict the class labels. It is a correct form as long as, for every pair of models $\model, \model'$ we have that $P_\model(x_1, \ldots, x_\ndata) = P_{\model'}(x_1, \ldots, x_\ndata)$. This assumption can be easily satisfied without specifying a particular model for the $x$.}
    \only<2>{
      If not dealing with time-series data, we assume independence between $x_t$:
      \[
      P_\model(y_1, \ldots, y_\ndata \mid  x_1, \ldots, x_\ndata)
      = \prod_{i=1}^T P_\model(y_i \mid x_i)
      \]
    }
  \end{block}
  \uncover<3->{
    \begin{block}{The \alert{Bayes rule} for maximising $\E_\bel(\util \mid a, x_t, \Training)$}
      The decision rule simply chooses the action:
      \begin{align}
        \decision_t &\in
                      \argmax_{\decision \in \Decision}
                      \sum_{y}  \alert<4>{\sum_{\model \in
                      \Model}  P_\model(y_t = y \mid x_t) \bel(\model \mid
                      \Training)} 
                      \util(\decision, y)
                      \only<5>{
        \\ &=
             \argmax_{\decision \in \Decision}
             \sum_{y} \Pr_{\bel \mid \Training}(y_t \mid x_t) 
             \util(\decision, y)
             }
      \end{align}
    \end{block}
  }
  \uncover<4->{
    We can rewrite this by calculating the posterior marginal marginal label probability
    \[
    \Pr_{\bel \mid \Training}(y_t \mid x_t) \defn
    \Pr_{\bel}(y_t \mid x_t, \Training) = 
    \sum_{\model \in \Model} P_\model(y_t \mid x_t) \bel(\model \mid \Training).
    \]
  }

\end{frame}

\begin{frame}
  \frametitle{Approximating the model}
  \begin{block}{Full Bayesian approach for infinite $\Model$}
    Here $\bel$ can be a probability density function and 
    \[
    \bel(\model \mid \Training)  = P_\model(\Training)  \bel(\model)  / \Pr_\bel(\Training),
    \qquad
    \Pr_\bel(\Training) = \int_{\Model} P_\model(\Training)  \bel(\model)  \dd,
    \]
    can be hard to calculate.
  \end{block}
  \onslide<2->{
    \begin{block}{Maximum a posteriori model}
      We only choose a single model through the following optimisation:
      \[
      \MAP(\bel, \Training) 
      \only<2>{
        = \argmax_{\model \in \Model} P_\model(\Training)  \bel(\model) 
      }
      \only<3>{
        = \argmax_{\model \in \Model}
        \overbrace{\ln P_\model(\Training)}^{\textrm{goodness of fit}}  + \underbrace{\ln \bel(\model)}_{\textrm{regulariser}}.
      }
      \]
      \only<article>{You can think of the goodness of fit as how well the model fits the training data, while the regulariser term simply weighs models according to some criterion. Typically, lower weights are used for more complex models.}
    \end{block}
  }
\end{frame}



\begin{frame}
  \frametitle{Learning outcomes}
  \begin{block}{Understanding}
    \begin{itemize}
    \item Preferences, utilities and the expected utility principle.
    \item Hypothesis testing and classification as decision problems.
    \item How to interpret $p$-values Bayesian tests.
    \item The MAP approximation to full Bayesian inference.
    \end{itemize}
  \end{block}
  
  \begin{block}{Skills}
    \begin{itemize}
    \item Being able to implement an optimal decision rule for a given utility and probability.
    \item Being able to construct a simple null hypothesis test.
    \end{itemize}
  \end{block}

  \begin{block}{Reflection}
    \begin{itemize}
    \item When would expected utility maximisation not be a good idea?
    \item What does a $p$ value represent when you see it in a paper?
    \item Can we prevent high false discovery rates when using $p$ values?
    \item When is the MAP approximation good?
    \end{itemize}
  \end{block}
  
\end{frame}



%%% Local Variables:
%%% mode: latex
%%% TeX-master: "notes"
%%% End:

 % decision hierarchies
\section{Classification with stochastic gradient descent}
\only<presentation>{
  \begin{frame}
    \tableofcontents[ 
    currentsection, 
    hideothersubsections, 
    sectionstyle=show/shaded
    ] 
  \end{frame}
}

\begin{frame}
  \frametitle{Classification as an optimisation problem.}
  \only<article>{Finding the optimal policy for our belief $\bel$ is not normally very difficult. However, it requires that we maintain the complete distribution $\bel$ and that we also under some probability distribution $P$. In simple decision problems, e.g. where the set of actions $\CA$ is finite, it is possible to do this calculation on-the-fly. However, in some cases we might not have a model.
    
Recall that we wish to maximise expected utility for some policy under some distribution. In general, this has the form 
\[
\max_\pol \E^\pol_\model(U).
\]
We also know that any expectation can be approximated by sampling. Let $P_\model(\outcome)$ be the distribution on outcomes defined by our model. Then
\[
\E^\pol_\model(U) = \sum_\omega U(a,\outcome) P_\model(\outcome)
\approx T^{-1} \sum_{t=1}^T U(a,\outcome_t), \qquad \outcome_t \sim P_\model(\outcome),
\]
i.e. when we can replace the explicit summation over all possible outcomes, weighed by their probability through averaging over $T$ outcomes sampled from the correct distribution. In fact this approximation is \alert{unbiased}, as its expectation is equal to the expected utility.
}

  \begin{block}{The $\model$-optimal classifier}
    \only<article>{Since the performance measure is simply an expectation, it is intuitive to directly optimise the decision rule with respect to an approximation of the expectation}
    \begin{align}
      \max_{\alert{\param} \in \Param} &f(\pol_{\alert{\param}}, \model, \util),
      &f(\pol_{\alert{\param}}, \model, \util) &\defn \E^{\pol_{\alert{\param}}}_\model(\util)\\
      f(\pol_{\alert{\param}}, \model, \util) &= 
      \sum_{x, y, a} \util(a, y) \pol_{\alert{\param}}(a \mid x) P_\model(y \mid x) P_\model(x)\\
      &\approx
        \sum_{t=1}^T \sum_{a_t} \util(a_t, y_t) \pol_{\alert{\param```232}}(a_t \mid x_t ),
        & (x_t, y_t)_{t=1}^T &\sim P_\model.
    \end{align}
    \only<article>{In practice, this is the empirical expectation on the training set $\cset{(x_t, y_t)}{t=1, \ldots, T}$. However, when the amount of data is insufficient, this expectation may be far from reality, and so our classification rule might be far from optimal.}
  \end{block}



  \only<article>{
  \begin{block}{The Bayes-optimal classifier}
An alternative idea is to use our uncertainty to create a distribution over models, and then use this distribution to obtain a single classifier that does take the uncertainty into account.
    \[
      \max_\Param f(\pol_\Param, \bel)
      \approx
      \max_\Param N^{-1} \sum_{n=1}^N
      \pol(a_t = y_n \mid x_t = x_n),
      \qquad
      (x_n, y_n) \sim P_{\model_n}, \model_n \sim \bel.
    \]
    In this case, the integrals are replaced by sampling models $\model_n$ from the belief, and then sampling $(x_n, y_n)$ pairs from $P_{\model_n}$.
  \end{block}
  }
\end{frame}

\only<presentation>{
 \againframe{beta-example}
}

\begin{frame}
  \frametitle{Stochastic gradient methdos}
  \only<article>{To find the maximum of a differentiable function $g$, we can use gradient descent}
  \begin{block}{Gradient ascent}
    \[
      \param_{i+1} = \param_i + \alpha \nabla_\param g(\param_i).
    \]
  \end{block}

  \only<article>{When $f$ is an expectation, we don't need to calculate the full gradient. In fact, we only need to take one sample from the related distribution.}
  \begin{block}{Stochastic gradient ascent}
    \[
      g(\param) = \int_\Model f(\param, \model) \dd \bel(\model)
    \]
    \[
      \param_{i+1} = \param_i + \alpha \nabla_\param f(\param_i, \model_i), \qquad \model_i \sim \bel.
    \]
  \end{block}
  \only<article>{Stochastic gradient methods are commonly employed in neural networks.} 
  
\end{frame}


\subsection{Neural network models}
\begin{frame}
  \frametitle{Two views of neural networks}
  \only<article>{In the simplest sense a neural network is simply as parametrised functions $f_\param$. In classification, neural networks can be used as  probabilistic models, so they describes the probability $P_\param(y | \bx)$, or as classification policies so that $f_\param(x, a)$ describes the probability $\pol_\param(a \mid x)$ of selecting class label $a$. Let us begin by describing the simplest type of neural network model, the perceptron.}
  
  \begin{block}{Neural network classification model $P_\param(\by \mid \bx_t)$}
    \begin{tikzpicture}
      \node[RV] at (0,0) (input) {$\bx_t$};
      \node[RV] at (4,0) (output) {$\by_t$};
      \draw[->, bend right] (input) to (output);
    \end{tikzpicture}    
    Objective: Find the best model for $\Training$.
  \end{block}

  
  \begin{block}{Neural network classification policy $\pol(a_t \mid \bx_t)$}
    \begin{tikzpicture}
      \node[RV] at (0,0) (input) {$\bx_t$};
      \node[RV] at (4,0) (output) {$a_t$};
      \draw[->, bend right] (input) to (output);
    \end{tikzpicture}    
    \vspace{1em}
    Objective: Find the best policy for $\util(a, \bx)$.
  \end{block}

  \uncover<2->{
    \begin{alertblock}{Difference between the two views}
      \begin{itemize}
      \item We can use standard probabilistic methods for $P$.
      \item Finding the optimal $\pol$ is an optimisation problem. \only<article>{However, estimating $P$ can also be formulated as an optimisation.}
      \end{itemize}
    \end{alertblock}
  }
\end{frame}


\begin{frame}
  \frametitle{Linear networks and the perceptron algorithm}
  \only<1,2>{
  \begin{figure}[H]
    \centering
    \begin{tikzpicture}
      \node[RV] at (0,0) (input) {$\bx$};
      \node[RV] at (4,0) (output) {$\ba$};
      \draw[->, bend right, label=above:$P$] (input) to (output);
      \uncover<2>{
      \node[RV, hidden] at (2,0) (param) {$\vparam$};
      \draw[->] (param) to (output);
      }
    \end{tikzpicture}
    \caption{Abstract graphical model for a neural network}
    \label{fig:gm-ann}
  \end{figure}
  }
  \only<article>{A neural network as used for modelling classification or regression problems, is simply a parametrised mapping $\CX \to \CY$. If we include the network parameters, then it is instead a mapping $\CX \times \Param \to \CY$, as seen in Figure~\ref{fig:gm-ann}.}
  \only<3,4>{
  \begin{figure}[H]
    \centering
    \begin{tikzpicture}
      \node[RV] at (0,-1) (input1) {$x_1$};
      \node[RV] at (0,1) (input2) {$x_2$};
      \node[RV] at (4,-1) (output1) {$a_1$};
      \node[RV] at (4,1) (output2) {$a_2$};
      \draw[->] (input1) to (output1);
      \draw[->] (input1) to (output2);
      \draw[->] (input2) to (output1);
      \draw[->] (input2) to (output2);
      \uncover<4>{
        \node[RV, hidden] at (2,-1.5) (param11) {$\param_{11}$};
        \node[RV, hidden] at (2,-0.5) (param12) {$\param_{12}$};
        \node[RV, hidden] at (2,0.5) (param21) {$\param_{211}$};
        \node[RV, hidden] at (2,1.5) (param22) {$\param_{22}$};
        \draw[->] (param22) to (output2);
        \draw[->] (param21) to (output2);
        \draw[->] (param12) to (output1);
        \draw[->] (param11) to (output1);
      }
      \draw[-, dashed] (input1) to (input2);
      \draw[-] (output1) to (output2);
    \end{tikzpicture}
    \caption{Graphical model for a linear neural network.}
    \label{fig:gm-ann}
  \end{figure}  
  \only<article>{If we see each possible output as a different random variable, this creates a dependence. After all, we are really splitting one variable into many. In particular, if the network's output is the probability of each action, then we must make sure these sum to 1.}
  }
  \only<5>{
  \begin{figure}[H]
    \centering
    \begin{tikzpicture}
      \node[RV] at (0,-1) (input1) {$x_1$};
      \node[RV] at (0,1) (input2) {$x_2$};
      \node[RV,label={$h_1(\bz) = e^{z_1} / [e^{z_1} + e^{z_2}]$}] at (8,-1) (output1) {$a_1$};
      \node[RV,label={$h_2(\bz) = e^{z_2} / [e^{z_1} + e^{z_2}]$}] at (8,1) (output2) {$a_2$};
      \node[RV,label={$g_{\vparam_1}(\bx) = \bx^\top \vparam_1$}] at (4,-1) (z1) {$z_1$};
      \node[RV,label={$g_{\vparam_2}(\bx) = \bx^\top \vparam_2$}] at (4,1) (z2) {$z_2$};
      \draw[->] (input1) to node[near end, below]{$\param_{11}$} (z1);
      \draw[->] (input1) to node[near end, below]{$\param_{12}$} (z2);
      \draw[->] (input2) to node[near end, above]{$\param_{21}$} (z1);
      \draw[->] (input2) to node[near end, above]{$\param_{22}$} (z2);      
      \draw[->] (z1) to (output1);
      \draw[->] (z1) to (output2);
      \draw[->] (z2) to (output1);
      \draw[->] (z2) to (output2);
    \end{tikzpicture}
    \caption{Architectural view of a linear neural network.}
    \label{fig:gm-ann}
  \end{figure}  
  }



  \begin{definition}[Linear classifier]
    \only<article>{
    A linear classifier with $\nobservations$ inputs and $\nclasses$ outputs is parametrised by}
    \[
    \mparam = 
    \begin{bmatrix}
      \vparam_{1} & \cdots & \vparam_{\nclasses}
    \end{bmatrix}
    =
    \begin{bmatrix}
      \param_{1,1} & \cdots & \param_{1,\nclasses}\\
      \vdots & \ddots & \vdots \\
      \param_{\nobservations} & \cdots & \param_{\nobservations,\nclasses}
    \end{bmatrix}
    \]
    \[
      \pol_\Param(a \mid \bx) = \exp\left(\vparam_a^\top \bx\right) / \sum_{a'} \exp\left(\vparam_{a'}^\top \bx\right) 
    \]
  \end{definition}
  \only<article>{Even though the classifier has a linear structure, the final non-linearity at the end is there to ensure that it defines a proper probability distribution over decisions.}
\only<article>{
  For classification problems, the observations $\bx_{t}$ are features $\bx_t = (x_{t,1} \ldots, x_{t,n})$ so that $\CX \subset \Reals^\nobservations$. 
It is convenient to consider the network output as a vector on the simplex $\by \in \Simplex^\nactions$, i.e. $\sum_{i=1}^\nclasses y_{t,i}  = 1$, $y_{t,i} \geq 0$. In the neural network model for classification, we typically ignore dependencies between the $x_{t,i}$ features, as we are not very interested in the distribution of $\bx$ itself.}

\end{frame}

\begin{frame}
\frametitle{Gradient ascent for a matrix $U$}
\begin{align}
  &\max_\param \sum_{t=1}^T \sum_{a_t} \util(a_t, y_t) \pol_\param(a_t \mid x_t ) \tag{objective}\\
  &\nabla_\param \sum_{t=1}^T \sum_{a_t} \util(a_t, y_t) \pol_\param(a_t \mid x_t ) \tag{gradient}\\
  =& \sum_{t=1}^T \sum_{a_t} \util(a_t, y_t) \nabla_\param \pol_\param(a_t \mid x_t )
\end{align}
\only<article>{We now need to calculate the gradient of the policy.}
\begin{block}{Chain Rule of Differentiation}
  \begin{align*}
    f(z), z = g(x), && \frac{df}{dx} &= \frac{df}{dg} \frac{dg}{dx} \tag{scalar version}\\
    && \nabla_\param \pol &= \nabla_g \pol \nabla_\param g \tag{vector version}
  \end{align*}
\end{block}
\end{frame}


\begin{frame}
  \frametitle{Learning outcomes}
  \begin{block}{Understanding}
    \begin{itemize}
    \item Classification as an optimisation problem.
    \item (Stochastic) gradient methods and the chain rule.
    \item Neural networks as probability models or classification policies.
    \item Linear neural netwoks.
    \item Nonlinear network architectures.
    \end{itemize}
  \end{block}
  
  \begin{block}{Skills}
    \begin{itemize}
    \item Using a standard NN class in python.
    \end{itemize}
  \end{block}

  \begin{block}{Reflection}
    \begin{itemize}
    \item How useful is the ability to have multiple non-linear layers in a neural network.
    \item How rich is the representational power of neural networks?
    \item Is there anything special about neural networks other than their allusions to biology?
    \end{itemize}
  \end{block}
  
\end{frame}


%%% Local Variables:
%%% mode: latex
%%% TeX-master: "notes"
%%% End:

 % linear models and stochastic gradient descent
\include{naive-bayes} % naive Bayes classifiers
%\section{Project: Credit risk for mortgages}

Consider a bank that must design a decision rule for giving loans to individuals. In this particular case, some of each individual's characteristics are partially known to the bank.  We can assume that the insurer has a linear utility for money and wishes to maximise expected utility. Assume that the $t$-th individual is associated with relevant information $x_t$, sensitive information $z_t$ and a potential outcome $y_t$, which is whether or not they will default on their mortgage. For each individual $t$, the decision rule chooses $a \in \CA$ with probability $\pol(a_t = a \mid x_t)$.

As an example, take a look at the historical data in \texttt{data/credit/german.data-mumeric}, described in \texttt{data/credit/german.doc}. Here there are some attributes related to financial situation, as well as some attributes related to personal information such as gender and marital status. 

A skeleton for the project is available at \url{https://github.com/olethrosdc/ml-society-science/tree/master/src/project-1}. Start with \verb|random_banker.py| as a template, and create a new module \verb|name_banker.py|. You can test your implementation with the \verb|TestLending.py| program. 

For ensuring progress, the project is split into two parts:
\subsection{Deadline 1: September 14}
The first part of the project focuses on a baseline implementation of a banker module.
\begin{enumerate}
\item Design a policy for giving or denying credit to individuals, given their probability for being credit-worthy. Assuming that if an individual is credit-worthy, you will obtain a return on investement of $r = 0.5\%$ per month. Take into account the length of the loan to calculate the utility through \verb|NameBanker.expected_utility()|. Assume that the loan is either fully repaid at the end of the lending period $n$, or not at all to make things simple. If an individual is not credit-worthy you will lose your investment of $m$ credits, otherwise you will gain $m [(1 + r)^{n} - 1]$ . Ignore macroenomic aspects, such as inflation. In this section, simply assume you have a model for predicting creditworthiness as input to your policy, which you can access \verb|NameBanker.get_proba()|. 
\item Implement \verb|NameBanker.fit()| to fit a model for calculating the probability of credit-worthiness from the german data. Then implement \verb|NameBanker.predict_proba()| to predict the probability of the loan being returned for new data. What are the implicit assumptions about the labelling process in the original data, i.e. what do the labels represent?
\item Combine the model with the first policy to obtain a policy for giving credit, given only the information about the individual and previous data seen. In other words, implement \verb|Namebanker.get_best_action()|.
\item Finally, using \verb|TestLending.py| as a baseline, create a jupyter notebook where you document your model development. Then compare your model against \verb|RandomBanker|.
\end{enumerate}

\subsection{Deadline 2: September 28}
The second part of the project focuses on issues of reproducibility, reliability, privacy and fairness. That is, how desirable would it be to use this model in practice? Here are some sample questions that you can explore, but you should be free to think about other questions.
\begin{enumerate}
\item Is it possible to ensure that your policy maximises revenue? How can you take into account the uncertainty due to the limited and/or biased data? What if you have to decide for credit for thousands of individuals and your model is wrong? How should you take that type of risk into account?\footnote{You do not need to implement anything specific for this to pass the assignment, but you should outline an algorithm in a precise enough manner that it can be implemented. In either case you should explain how your solution mitigates this type of risk.}
\item Does the existence of this database raise any privacy concerns? If the database was secret (and only known by the bank), but the credit decisions were public, how would that affect privacy? (a) Explain how you would protect the data of the people in the training set. (b) Explain how would protect the data of the people that apply for new loans. (c) \emph{Implement} a private decision making mechanism for (b),\footnote{If you have already implemented (a) as part of the tutorial, feel free to include the results in your report.} and estimate the amount of loss in utility as you change the privacy guarantee.
\item Choose one concept of fairness, e.g. balance of decisions with respect to gender. How can you ensure that your policy is fair? How can you measure it? How does the original training data affect the fairness of your policy? \footnote{You do not need to implement any type of fair policy a passing grade, but you should at least try to analyse the data or your decision function with simple statistics.}
\end{enumerate}

Submit a final report about your project, either as a standalone PDF or as a jupyter notebook.

%%% Local Variables:
%%% mode: latex
%%% TeX-master: notes
%%% End:


\chapter{Privacy}
\only<article>{
  Participating in a study always carries a risk for individuals, namely that of data disclosure. In this chapter, we first explain how simple database query methods, and show even a small number of queries to a database they can compromise the privacy of individuals. We then introduce to formal concepts of privacy protection: $k$-anonymity and differential privacy. The first is relatively simple to apply and provides some limited resistance to identification of individuals through record linkage attacks. The latter is a more general concept, and can be simple apply in some settings, while it offers information-theoretic protection to individuals. A major problem with any privacy definition and method, however is correct interpretation of the privacy concept used, and correct implementation of the algorithm used.}

\section{Database access models}
\only<presentation>{
  \begin{frame}
    \tableofcontents[ 
    currentsection, 
    hideothersubsections, 
    sectionstyle=show/shaded
    ] 
  \end{frame}
}

\begin{frame}
  \frametitle{Databases}
  \begin{example}[Typical relational database in a tax office]
    \begin{table}[H]
      \centering
  \begin{tabular}{l|l|l|l|l|l|l}
    ID & Name &  Salary & Deposits & Age & Postcode & Profession\\
    \hline
    1959060783 & Mike Pence & 150,000 & 1e6 & 60 & 1001 & Politician\\
    1946061408 & Donald Trump & 300,000 & -1e9 & 72 & 1001 & Rentier\\
    2100010101 & A. B. Student & 10,000 & 100,000 & 40 & 1001 & Time Traveller
  \end{tabular}
\end{table}
\end{example}

\only<1>{
  \begin{block}{Database access}
    \begin{itemize}
    \item When owning the database: Direct look-up.
    \item When accessing a server etc: Query model.
    \end{itemize}
  \end{block}
}
\only<2>{
  \begin{figure}[H]
    \centering
    \begin{tikzpicture}
        \node[rectangle] at (0,0) (python) {Python program};
        \node[rectangle] at (8,0) (database) {Database System};
        \draw[thickarrow, bend right]   (python) to node[black]{Query} (database) ;
        \draw[thickarrow, bend right]   (database) to node[black]{response} (python) ;
      \end{tikzpicture}
    \label{fig:database-access}
    \caption{Database access model}
  \end{figure}
}
  
\end{frame}

\begin{frame}
  \frametitle{Queries in SQL}
  \begin{block}{The \texttt{SELECT} statement}
    \begin{itemize}
    \item \texttt{SELECT column1, column2 FROM table;}
      \only<article>{This selects only some columns from the table}
    \item \texttt{SELECT * FROM table;}
      \only<article>{This selects all the columns from the table}
    \end{itemize}
  \end{block}

  \begin{block}{Selecting rows}
    \texttt{SELECT * FROM table WHERE column = value;}
  \end{block}

  \begin{exampleblock}{Arithmetic queries}
    \only<article>{Here are some example SQL statements}
    \begin{itemize}
    \item  \texttt{SELECT COUNT(column) FROM table WHERE condition;}
      \only<article>{This allows you to count the number of rows matching \texttt{condition}}
    \item  \texttt{SELECT AVG(column) FROM table WHERE condition;}
      \only<article>{This lets you to count the number of rows matching \texttt{condition}}
    \item  \texttt{SELECT SUM(column) FROM table WHERE condition;}
      \only<article>{This is used to sum up the values in a column.}
    \end{itemize}
  \end{exampleblock}

\end{frame}



%%% Local Variables:
%%% mode: latex
%%% TeX-master: "notes"
%%% End:
 % data base access model


\section{Privacy in databases}
\only<presentation>{
  \begin{frame}
    \tableofcontents[ 
    currentsection, 
    hideothersubsections, 
    sectionstyle=show/shaded
    ] 
  \end{frame}
}

\begin{frame}
  \frametitle{Anonymisation}
  \only<article>{If we wish to publish a database, frequently we need to protect identities of people involved. The simplest method for doing that is simply erasing directly identifying information. However, this does not really work most of the time, especially since attackers can have side-information that can reveal the identities of individuals in the original data.}
  
  \begin{example}[Typical relational database in Tinder]
    \begin{table}[H]
      \begin{tabular}{l|l|l|l|l|l|l}
        Birthday & Name & Height  & Weight & Age & Postcode & Profession\\
        \hline
        06/07 & \only<1>{Li Pu} & 190 & 80 & 60-70 & 1001 & Politician\\
        06/14 & \only<1>{Sara Lee} & 185 & 110 & 70+ & 1001 & Rentier\\
        01/01 & \only<1>{A. B. Student} & 170 & 70 & 40-60 & 6732 & Time Traveller
      \end{tabular} 
    \end{table}
  \end{example}

  \only<2>{The simple act of hiding or using random identifiers is called anonymisation.}
  \only<article>{However this is generally insufficient as other identifying information may be used to re-identify individuals in the data.}
\end{frame}


\begin{frame}
  \frametitle{Record linkage}
  \only<article>{In particular, anonymisation is not enough as record linkage can allow you to still extract information using data from another database through \emph{quasi-identifiers}.}

  \only<1>{
    \def\firstcircle{(0,0) circle (7em)}
    \def\secondcircle{(3,0) circle (7em)}
    
    \begin{figure}[H]
      \centering
      \begin{tikzpicture}

        \draw \firstcircle node[text width=7em] {Ethnicity\newline
          Date\newline Diagnosis \newline Procedure \newline
          Medication \newline Charge }; \draw \secondcircle node [text
        width=2em, align=right] {Name \newline Address \newline
          Registration \newline Party \newline Lastvote};
        \begin{scope}
          \clip \firstcircle; \fill[red] \secondcircle;
        \end{scope}
        \node [text width=4em] (QI) at (1.5, 0) {Postcode \newline
          Birthdate \newline Sex}; \node [text width=4em] (qi-text) at
        (1.5, -3) {Quasi-identifiers}; \path[->]<1-> (qi-text) edge
        [bend left] (QI);
        % Now we want to highlight the intersection of the first and
        % the
        % second circle:


      \end{tikzpicture}
      
      \caption{An example of two datasets, one containing sensitive and the other public information. The two datasets can be linked and individuals identified through the use of quasi-identifiers.}
      \label{fig:quasi-identifiers}
    \end{figure}
  }
  
  \begin{example}[Typical relational database in a tax office]
    \begin{table}[H]
      \begin{tabular}{l|l|l|l|l|l|l}
        ID & Name &  Salary & Deposits & Age & Postcode & Profession\\
        \hline
        1959060783 & Li Pu & 150,000 & 1e6 & 60 & 1001 & Politician\\
        1946061408 & Sara Lee & 300,000 & -1e9 & 72 & 1001 & Rentier\\
        2100010101 & A. B. Student & 10,000 & 100,000 & 40 & 6732 & Time Traveller
      \end{tabular}
    \end{table}
  \end{example}
  
  \begin{example}[Typical relational database in Tinder]
    \begin{table}[H]
      \begin{tabular}{l|l|l|l|l|l|l}
        Birthday & Name & Height  & Weight & Age & Postcode & Profession\\
        \hline
        06/07 & & 190 & 80 & 60-70 & 1001 & Politician\\
        06/14 &  & 185 & 110 & 70+ & 1001 & Rentier\\
        01/01 &  & 170 & 70 & 40-60 & 6732 & Time Traveller
      \end{tabular}
    \end{table}
  \end{example}
\end{frame}

% \begin{frame}
%   \frametitle{Data linkage with SQL}
%     The original database \verb|database| and adversary side information \verb|side-information| can be combined using the following simple SQL query:
% \begin{verbatim}
% SELECT * FROM database JOIN side-information ON [condition]
% \end{verbatim}
% where \verb|condition| describes how to match the records.

%   \begin{example}
%     For the databases given above, we could use
% \begin{verbatim}
% SELECT * FROM tinder JOIN tax ON tinder.height = tax.height AND tinder.age = tax.age
% \end{verbatim}
%     to create a joint table.
%   \end{example}
% \end{frame}

\section{$k$-anonymity}

\begin{frame}
  \frametitle{$k$-anonymity}
  \begin{figure}[H]
    \centering \subfigure[Samarati]{\includegraphics[width=0.25\textwidth]{../figures/samarati}}
    \subfigure[Sweeney]{\includegraphics[width=0.25\textwidth]{../figures/sweeney}}
  \end{figure}
  \only<article>{The concept of $k$-anonymity was introduced by~\citet{samarati1998protecting} and provides good guarantees when accessing a single database}

  \begin{definition}[$k$-anonymity]
    A database provides $k$-anonymity if for every person in the database is indistinguishable from $k-1$ persons with respect to \emph{quasi-identifiers}.
  \end{definition}
  \alert{It's the analyst's job to define quasi-identifiers}
  
\end{frame}

\begin{frame}
  \only<1>{
    \begin{table}[H]
      \begin{tabular}{l|l|l|l|l|l|l}
        Birthday & Name & Height  & Weight & Age & Postcode & Profession\\
        \hline
        06/07 & Li Pu & 190 & 80 & 60+ & 1001 & Politician\\
        06/14 & Sara Lee & 185 & 110 & 60+ & 1001 & Rentier\\
        06/12 & Nikos Papadopoulos & 170 & 82 & 60+ & 1243 & Politician\\
        01/01 & A. B. Student & 170 & 70 & 40-60 & 6732 & Time Traveller\\
        05/08 & Li Yang & 175 & 72 & 30-40 & 6910 & Time Traveller
      \end{tabular}
      \caption{1-anonymity.}
    \end{table}

  }
  \only<presentation>{
    \only<2>{
      \begin{tabular}{l|l|l|l|l|l|l}
        Birthday & Name & Height  & Weight & Age & Postcode & Profession\\
        \hline
        06/07 &  & 190 & 80 & 60+ & 1001 & Politician\\
        06/14 &  & 185 & 110 & 60+ & 1001 & Rentier\\
        06/12 &  & 170 & 82 & 60+ & 1243 & Politician\\
        01/01 &  & 170 & 70 & 40-60 & 6732 & Time Traveller\\
        05/08 &  & 175 & 72 & 30-40 & 6910 & Policeman
      \end{tabular}
      1-anonymity
    }

    \only<3>{
      \begin{tabular}{l|l|l|l|l|l|l}
        Birthday & Name & Height  & Weight & Age & Postcode & Profession\\
        \hline
        06/07 &  & 180-190 & 80+ & 60+ & 1* & \\
        06/14 &  & 180-190 & 80+ & 60+ & 1* &\\
        06/12 &  & 170-180 & 60+ & 60+ & 1* & \\
        01/01 &  & 170-180 & 60-80 & 20-60 & 6* &\\
        05/08 &  & 170-180 & 60-80 & 20-60 & 6* & 
      \end{tabular}
      1-anonymity
    }
  }
  \only<4>{
    \begin{table}[H]
      \begin{tabular}{l|l|l|l|l|l|l}
        Birthday & Name & Height  & Weight & Age & Postcode & Profession\\
        \hline
                 &  & 180-190 & 80+ & 60+ & 1* & \\
                 &  & 180-190 & 80+ & 60+ & 1* &\\
                 &  & 170-180 & 60-80 & 69+ & 1* & \\
                 &  & 170-180 & 60-80 & 20-60 & 6* &\\
                 &  & 170-180 & 60-80 & 20-60 & 6* & 
      \end{tabular}
      \caption{2-anonymity: the database can be partitioned in sets of at least 2 records}
    \end{table}
  }

  \only<article>{However, with enough information, somebody may still be able to infer something about the individuals}
\end{frame}





\section{Differential privacy}
\only<article>{While $k$-anonymity can protect against specific re-identification attacks when used with care, it says little about what to do when the adversary has a lot of power. For example, if the  adversary knows the data of everybody that has participated in the database,  it is trivial for them to infer what our own data is. Differential privacy offers protection against adversaries with unlimited side-information or computational power. Informally, an algorithmic computation is differentially-private if an adversary cannot distinguish two similar database based on the result of the computation. While the notion of similarity is for the analyst to define, it is common to say that two databases are similar when they are identical apart from the data of one person.}

\begin{frame}
  \begin{figure}[H]
    \begin{tikzpicture}
      \node[label=left:$x$] at (0,0) (data) {\includegraphics[width=0.2\columnwidth]{../figures/medical}};

      \node[label=$x_1$] at (-2,3)(patient1) {\includegraphics[width=0.1\columnwidth]{../figures/me-recent}};
      \uncover<3->{
        \node[label=$x_2$] at (2,3) (patient2) {\includegraphics[width=0.2\columnwidth]{../figures/judge}};
      }
      \uncover<4->{
        \node[label=$a$] at (4,0)   (statistics) {\includegraphics[width=0.3\columnwidth]{../figures/coronary-disease}};
      }
      \uncover<2->{
        \draw[->] (patient1) -- (data);
      }
      \uncover<3->{
        \draw[->] (patient2) -- (data);
      }
      \uncover<4->{
        \draw[->] (data) -- node[above]{$\pol$} (statistics);
      }
      \uncover<5->{
        \draw[line width=5, red, ->] (statistics) -- (patient2);
      }
    \end{tikzpicture}
    \caption{If two people contribute their data $x = (x_1, x_2)$ to a medical database, and an algorithm $\pol$ computes some public output $a$ from $x$, then it should be hard infer anything about the data from the public output.}
  \end{figure}

\end{frame}

\begin{frame}
  \frametitle{Privacy desiderata}
  \only<article>{
    Consider a scenario where $n$ persons give their data $x_1, \ldots, x_n$ to an analyst. This analyst then performs some calculation $f(x)$ on the data and published the result. The following properties are desirable from a general standpoint.

    \paragraph{Anonymity.} Individual participation in the study remains a secret. From the release of the calculations results, nobody can significantly increase their probability of identifying an individual in the database.

    \paragraph{Secrecy.} The data of individuals is not revealed. The release does not significantly increase the probability of inferring individual's information $x_i$.

    \paragraph{Side-information.} Even if an adversary has arbitrary side-information, he cannot use that to amplify the amount of knowledge he would have obtained from the release.

    \paragraph{Utility.} The released result has, with high probability, only a small error relative to a calculation that does not attempt to safeguard privacy.
  }
  \only<presentation>{
    We wish to calculate something on some private data and publish a \alert{privacy-preserving}, but \alert{useful}, version of the result.
    \begin{itemize}
    \item Anonymity: Individual participation remains hidden.
    \item Secrecy: Individual data $x_i$ is not revealed.
    \item Side-information: Linkage attacks are not possible.
    \item Utility: The calculation remains useful.
    \end{itemize}
  }
\end{frame}

\begin{frame}
  \frametitle{Example: The prevalence of drug use in sport}
  
  \only<article>{
    Let's say you need to perform a statistical analysis of the drug-use habits of athletes. Obviously, even if you promise the athlete not to reveal their information, you still might not convince them. Yet, you'd like them to be truthful. The trick is to allow them to randomly change their answers, so that you can't be \emph{sure} if they take drugs, no matter what they answer.
  }

  \only<presentation>{
    \begin{itemize}
    \item $n$ athletes
    \item Ask whether they have doped in the past year.
    \item Aim: calculate \% of doping.
    \item How can we get truthful / accurate results?
    \end{itemize}
  }
  \only<2>{
    \begin{block}{Algorithm for randomising responses about drug use}
      \begin{enumerate}
      \item Flip a coin.
      \item If it comes heads, respond truthfully. 
      \item Otherwise, flip another coin and respond \texttt{yes} if it comes heads and \texttt{no} otherwise.
      \end{enumerate}
    \end{block}

    \begin{exercise}
      Assume that the observed rate of positive responses in a sample is $p$, that everybody follows the protocol, and the coin is fair. Then, what is the true rate $q$ of drug use in the population?
    \end{exercise}
  }
  \onslide<3->{
    \begin{proof}[Solution]
      Since the responses are random, we will deal with expectations first
        \begin{align*}
          \E p
          &= \frac{1}{2} \times \frac{1}{2} + q \times \frac{1}{2}
          \uncover<4->{= \frac{1}{4} + \frac{q}{2}}
          \uncover<5->{\\
          q &= 2 \E p - \frac{1}{2}.}
        \end{align*}
      
    \end{proof}
  }
  \only<article>{The problem with this approach, of course, is that we are effectively throwing away half of our data. In particular, if we repeated the experiment with a coin that came heads at a rate $\epsilon$, then our error bounds would scale as $O(1/\sqrt{\epsilon n})$ for $n$ data points.}
\end{frame}

\begin{frame}
  \frametitle{The randomised response mechanism}
  \only<article>{The above idea can be generalised. Consider we have data $x_1, \ldots, x_n$ from $n$ users and we transform it randomly to $y_1, \ldots, y_n$ using the following mapping.}
  \begin{definition}[Randomised response]
    The $i$-th user, whose data is $x_i \in \{0,1\}$ , responds with $a_i \in \{0, 1\}$ with probability
    \[
      \pol(a_i = j \mid x_i = k) = p,  \qquad  \pol(a_i = k \mid x_i = k) = 1 - p,
    \]
    where $j \neq k$.
  \end{definition}

  \uncover<2->{Given the complete data $x$, the mechanism's output is $a = (a_1, \ldots, a_n)$.}
  \uncover<3->{Since the algorithm independently calculates a new value for each data entry, the output is
    \[
      \pol(a \mid x) = \prod_i \pol(a_i \mid x_i)
    \]
  }

  \only<article>{This mechanism satisfies so-called $\epsilon$-differential privacy, which we will define later.}

\end{frame}

\begin{frame}
  \begin{exercise}
    Let the adversary have a prior $\bel(x = 0) = 1 - \bel(x = 1)$ over the values of the true response of an individual. we use the randomised response mechanism with $p$ and the adversary observes the randomised data $a = 1$ for that individual, then what is $\bel(x = 1 \mid a = 1)$?
  \end{exercise}
\end{frame}

\begin{frame}
  \frametitle{The local privacy model}
  \begin{figure}[H]
    \centering
    \begin{tikzpicture}
      \node[RV] at (0,0) (x1) {$x_1$};
      \node[RV] at (0,1) (x2) {$x_2$};
      \node[RV] at (0,2) (xn) {$x_n$};
      \node[RV] at (2,0) (a1) {$a_1$};
      \node[RV] at (2,1) (a2) {$a_2$};
      \node[RV] at (2,2) (an) {$a_n$};
      \draw[->] (x1) -- (a1);
      \draw[->] (x2) -- (a2);
      \draw[->] (xn) -- (an);
      % \node[select] at (1,-1) (pol) {$\pol$};
      % \draw[->] (pol) -- (a1);
      % \draw[->] (pol) -- (a2);
      % \draw[->] (pol) -- (an);
    \end{tikzpicture}
    
    \caption{The local privacy model}
    \label{fig:local-privacy}
  \end{figure}
  \only<article>{
    In the local privacy model, the $i$-th individual's data $x_i$ is used to generate a private response $a_i$. This means that no individual will provide their true data with certainty. This model allows us to publish a complete dataset of private responses.
    }
\end{frame}

\begin{frame}
  \frametitle{Differential privacy.}
  \includegraphics[width=0.2\textwidth]{../figures/dwork} \hspace{1em}
  \includegraphics[width=0.2\textwidth]{../figures/mcsherry} \hspace{1em}
  \includegraphics[width=0.2\textwidth]{../figures/nissim} \hspace{1em}
  \includegraphics[width=0.2\textwidth]{../figures/smith}
  \only<article>{Now let us take a look at a way to characterise the  the inherent privacy properties of algorithms. This is called differential privacy, and it can be seen as a bound on the information an adversary with arbitrary power or side-information could extract from the result of a computation $\pol$ on the data. For reasons that will be made clear later, this computation has to be stochastic.}
  
  \begin{definition}[$\epsilon$-Differential Privacy]
    A stochastic algorithm $\pol : \CX \to \CA$, where $\CX$ is endowed with a neighbourhood relation $N$, is said to be $\epsilon$-differentially private if
    \begin{equation}
      \label{eq:epsilon-dp}
      \left|\ln \frac{\pol(a \mid x)}{\pol(a \mid x')}\right| \leq \epsilon , \qquad \forall x N x'.
    \end{equation}
  \end{definition}
  
  \only<article>{Typically, algorithms are applied to datasets $x = (x_1, \ldots, x_n)$ composed of the data of $n$ individuals. Thus, all privacy guarantees relate to the data contributed by these individuals. 

    In this context, two datasets are usually called neighbouring if $x = (x_1, \ldots, x_{i-1}, x_i, x_{i+1} x_n)$ and 
    $x' = (x_1, \ldots, x_{i-1}, x_{i+1} x_n)$, i.e. if one dataset is missing an element.
    
    A slightly weaker definition of neighbourhood is to say that $x N x'$ if $x' = (x_1, \ldots, x_{i-1}, x'_i, x_{i+1} x_n)$, i.e. if one dataset has an altered element. We will usually employ this latter definition, especially for the local privacy model.

  }
\end{frame}

\begin{frame}
  \begin{block}{The definition of differential privacy}
    \begin{itemize}
    \item First rigorous mathematical definition of privacy.
    \item Relaxations and generalisations possible.
    \item Connection to learning theory and reproducibility.
    \end{itemize}
  \end{block}

  \begin{block}{Current uses}
    \begin{itemize}
    \item Apple. \only<article>{DP is used internally in the company to ``protect user privacy''. It is not clear exactly what they are doing but their efforts seem to be going in the right direction.}
    \item Google. \only<article>{The company has a DP API available based on randomised response, RAPPOR.}
    \item Uber. \only<article>{Elastic sensitivity for SQL queries, which is available as open source. This is a good thing, because it is easy to get things wrong with privacy.}
    \item US 2020 Census. \only<article>{It uses differential privacy to protect the condidentiality of responders' information while maintaining data that are suitable for their intended uses.}
    \end{itemize}
  \end{block}

  \begin{block}{Open problems}
    \begin{itemize}
    \item Complexity of differential privacy.
    \item Verification of implementations and queries.
    \end{itemize}
  \end{block}
\end{frame}

\begin{frame}
  \only<article>{
    \begin{remark}
      Any differentially private algorithm must be stochastic.
    \end{remark}

    To prove that this is necessary, consider the example of counting how many people take drugs in a competition. If the adversary only doesn't know whether you in particular take drugs, but knows whether everybody else takes drugs, it's trivial to discover your own drug habits by looking at the total. This is because in this case, $f(x) = \sum_i x_i$ and the adversary knows $x_i$ for all $i \neq j$. Then, by observing $f(x)$, he can recover $x_j = f(x) - \sum_{i \neq j} x_i$. Consequently, it is not possible to protect against adversaries with arbitrary side information without stochasticity.}
  \begin{remark}
    The randomised response mechanism with $p \leq 1/2$ is $(\ln \frac{1 - p}{p})$-DP.
  \end{remark}
  \begin{proof}
    Consider $x = (x_1, \ldots, x_j,  \ldots, x_n)$, $x' = (x_1, \ldots, x'_j,  \ldots, x_n)$. Then
    \begin{align*}
      \pol(a \mid x)
      \uncover<2->{&= \prod_i \pol(a_i \mid x_i)}
                     \uncover<3->{\\ &= \pol(a_j \mid x_j) \prod_{i \neq j} \pol(a_i \mid x_i) }
                                       \uncover<4->{\\ &\leq \frac{p}{1 - p} \pol(a_j \mid x'_j) \prod_{i \neq j} \pol(a_i \mid x_i) }
                                                         \uncover<5>{\\ &= \frac{1-p}{p} \pol(a \mid x')}
    \end{align*}
    \only<4>{$\pol(a_j = k\mid x_j = k) = 1 - p$ so the ratio is $\max\{(1-p)/p, p/(1 - p)\} \leq (1 - p)/p$ for $p \leq 1/2$.}
  \end{proof}
\end{frame}

\begin{frame}
  \begin{figure}[H]
    \centering
    \begin{tikzpicture}
      \node[rectangle] at (0,0) (python) {Python program};
      \node[rectangle] at (8,0) (database) {Database System};
      \draw[thickarrow, bend right]   (python) to node[black]{Query $q$} (database) ;
      \draw[thickarrow, bend right]   (database) to node[black]{Private response $a$} (python) ;
    \end{tikzpicture}
    \label{fig:database-access}
    \caption{Private database access model}
  \end{figure}
  \begin{block}{Response policy}
    The  policy defines a distribution over responses $a$ given the data $x$ and the query $q$.
    \[
      \pol(a \mid x, q)
    \]
  \end{block}
\end{frame}

\begin{frame}
  \frametitle{Differentially private queries}
  \only<article>{There is no actual \texttt{DP-SELECT} statement, but we can imagine it.}
  \begin{block}{The \texttt{DP-SELECT} statement}
    \begin{itemize}
    \item \texttt{DP-SELECT $\epsilon$ column1, column2 FROM table;}
      \only<article>{This selects only some columns from the table}
    \item \texttt{DP-SELECT $\epsilon$ * FROM table;}
      \only<article>{This selects all the columns from the table}
    \end{itemize}
  \end{block}

  \begin{block}{Selecting rows}
    \texttt{DP-SELECT $\epsilon$  * FROM table WHERE column = value;}
  \end{block}

  \begin{exampleblock}{Arithmetic queries}
    \only<article>{Here are some example SQL statements}
    \begin{itemize}
    \item  \texttt{DP-SELECT $\epsilon$ COUNT(column) FROM table WHERE condition;}
      \only<article>{This allows you to count the number of rows matching \texttt{condition}}
    \item  \texttt{DP-SELECT $\epsilon$ AVG(column) FROM table WHERE condition;}
      \only<article>{This lets you to count the number of rows matching \texttt{condition}}
    \item  \texttt{DP-SELECT $\epsilon$ SUM(column) FROM table WHERE condition;}
      \only<article>{This is used to sum up the values in a column.}
    \end{itemize}
  \end{exampleblock}

    \only<article>{Depending on the DP scheme, each query answered may leak privacy. In particular, if we always respond with an $\epsilon$-DP mechanism, after $T$ queries our privacy guarantee is $T \epsilon$. There exist mechanisms that do not respond to each query independently, which can bound the total privacy loss.

      \begin{definition}[$T$-fold adaptive composition]
        In this privacy model, an adversary is allowed to compose $T$ queries. The composition is \alert{adaptive}, in the sense that the next query is allowed to depend on the previous queries and their results.
      \end{definition}
      \begin{theorem}
        For any $\epsilon > 0$, the class of $\epsilon$-differentially private mechanism satisfy $T \epsilon$-differential privacy under $T$-fold adaptive composition.
      \end{theorem}
 }
    \only<presentation>{
      \begin{alertblock}{Composition}
        If we answer $T$ queries with an $\epsilon$-DP mechanism, then our cumulative privacy loss is $\epsilon T$.
      \end{alertblock}
    }
\end{frame}

\begin{frame}
  \begin{exercise}{Adversary knowledge}
    \only<article>{Assume that the adversary knows that the data is either $\bx$ or $\bx'$. For concreteness, assume the data is either }
    \[
      \bx = (x_1, \ldots, x_j = 0, \ldots,  x_n)
    \]
    \only<article>{where $x_i$ indicates whether or not the $i$-th person takes drugs, or}
    \[
      \bx' = (x_1, \ldots, x_j=1, \ldots, x_n).
    \]
    \only<article>{In other words, the adversary knows the data of all people apart from one, the $j$-th person. We can assume that the adversary has some prior belief}
    \[
      \bel(\bx) = 1 - \bel(\bx')
    \]
    \only<article>{for the two cases. Assume the adversary knows
      the output $a$ of a mechanism $\pol$}
    \only<presentation>{
      \onslide<2->{
        \[
          a_t, \qquad \pol(a_t \mid \bx_t) \Rightarrow
          \begin{cases}
            \pol(a_t \mid \bx_t = \bx)\\
            \pol(a_t \mid \bx_t = \bx')
          \end{cases}
        \]
      }
    }
    What can we say about the posterior distribution of the adversary $\bel(\bx \mid a, \pol)$ after having seen the output, if $\pol$ is $\epsilon$-DP?
  \end{exercise}
  
\end{frame}
\only<article>{
  \begin{frame}
    \begin{block}{Solution}
      We can write the adversary posterior as follows.
      \begin{align}
        \bel(\bx \mid a, \pol)
        &=
        \frac{\pol(a  \mid \bx) \bel(\bx)}
          {\pol(a  \mid \bx) \bel(\bx) + \pol(a  \mid \bx') \bel(\bx')}
        \\
        &\geq
          \frac{\pol(a  \mid \bx) \bel(\bx)}
          {\pol(a  \mid \bx) \bel(\bx) + \pol(a  \mid \bx) e^\epsilon \bel(\bx')} \tag{from DP definition}
        \\
        &=
          \frac{\bel(\bx)}
          {\bel(\bx) +  e^\epsilon \bel(\bx')}
      \end{align}
      But this is not very informative. We can also write
      \begin{align}
        \frac{\bel(\bx \mid a, \pol)}{\bel(\bx' \mid a, \pol)}
        =
        \frac{\pol(a  \mid \bx) \bel(\bx)}{\pol(a  \mid \bx') \bel(\bx')}
        \geq
        \frac{\pol(a  \mid \bx) \bel(\bx)}{\pol(a  \mid \bx) e^{-\epsilon} \bel(\bx')}
        =
        \frac{\bel(\bx)}{\bel(\bx')} e^\epsilon
      \end{align}
    \end{block}
  \end{frame}
}

\begin{frame}
  \frametitle{Dealing with multiple attributes.}

  \only<article>{Up to now we have been discussing the case where each individual only has one attribute. However, in general each individual $t$ contributes multiple data $x_{t.i}$, which can be considered as a row $\bx_t$ in a database. Then the mechanism can release each $a_{t,i}$ independently.}
  \begin{block}{Independent release of multiple attributes.}
    For $n$ users and $k$ attributes, if the release of each attribute $i$ is $\epsilon$-DP then 
    the data release is $k \epsilon$-DP. Thus to get $\epsilon$-DP overall,  we need $\epsilon / k$-DP per attribute.
  \end{block}
  \only<article>{The result follows immediately from the composition theorem. We can see each attribute release as the result of an individual query.}

\end{frame}
\subsection{Other differentially private mechanisms}
\begin{frame}
  \frametitle{The Laplace mechanism.}
  \only<article>{
    A simple method to obtain a differentially private algorithm from a deterministic function $f : \CX \to \Reals$, is to use additive noise, so that the output of the algorithm is simply 
    \[
      a = f(x) + \omega, \qquad \omega \sim \Laplace.
    \]
    The amount of noise added, together with the smoothness of the function $f$, determine the amount of privacy we have.
  }
  \begin{definition}[The Laplace mechanism]
    For any function $f : \CX \to \Reals$, 
    \begin{equation}
      \label{eq:laplace-mechanism}
      \pol(a \mid x) = \Laplace(f(x), \lambda),
    \end{equation}
    where the Laplace density is defined as
    \[
      p(\omega \mid \mu, \lambda) = \frac{1}{2 \lambda} \exp\left(-\frac{|\omega - \mu|}{\lambda}\right).
    \]
    and has mean $\mu$ and variance $2 \lambda^2$.
  \end{definition}
  \only<article>{Here, $\Laplace(\mu, \lambda)$ is the density $f(x) = \frac{\lambda}{2} \exp(-\lambda |x - \mu|)$}.
\end{frame}

\begin{frame}
  \begin{example}[Calculating the average salary]
    \begin{itemize}
    \item The $i$-th person receives salary $x_i$
    \item We wish to calculate the average salary in a private manner.
    \end{itemize}
  \end{example}
  \begin{block}{Local privacy model}
    \begin{itemize}
    \item Obtain $y_i = x_i + \omega$, where $\omega \sim \Laplace(\lambda)$.
    \item Return $a = n^{-1} \sum_{i=1}^n y_i$.
    \end{itemize}
  \end{block}
  \begin{block}{Centralised privacy model}
    Return $a = n^{-1} \sum_{i=1}^n x_i + \omega$, where $\omega \sim \Laplace(\lambda')$.
  \end{block}
  
  How should we add noise in order to guarantee privacy?
\end{frame}

\begin{frame}
  \frametitle{The centralised privacy model}
  \begin{figure}[H]
    \centering
    \begin{tikzpicture}
      \node[RV] at (0,0) (x1) {$x_1$};
      \node[RV] at (0,1) (x2) {$x_2$};
      \node[RV] at (0,2) (xn) {$x_n$};
      \node[RV] at (2,1) (a) {$a$};
      \node[select] at (1,-1) (pol) {$\pol$};
      \draw[->] (x1) -- (a);
      \draw[->] (x2) -- (a);
      \draw[->] (xn) -- (a);
      \draw[->] (pol) -- (a);
    \end{tikzpicture}
    
    \caption{The centralised privacy model}
    \label{fig:centralised-privacy}
  \end{figure}
  \begin{assumption}
    The data $x$ is collected and the result $a$ is published by a \alert{trusted curator}
  \end{assumption}
\end{frame}

\begin{frame}
  \frametitle{DP properties of the Laplace mechanism}
  \begin{definition}[Sensitivity]
    The sensitivity of a function $f$ is
    \[
      \sensitivity{f} \defn \sup_{x N x'} |f(x) - f(x')|
    \]
    \only<article>{
      If we define a metric $d$, so that $d(x, x') = 1$ for $x N x'$, then:
      \[
        |f(x) - f(x')| \leq \sensitivity{f} d(x, x'),
      \]
      i.e. $f$ is $\sensitivity{f}$-Lipschitz with respect to $d$.
    }
  \end{definition}
  \begin{example}
    If $f: \CX \to [0, B]$, e.g. $\CX = \Reals$ and $f(x) = \min\{B, \max\{0, x\}\}$, then
    \onslide<2->{
      $\sensitivity{f} = B$.
    }
  \end{example}
  \onslide<3->{
    \begin{example}
      If $f: [0,B]^n \to [0,B]$ is
      $f = \frac{1}{n} \sum_{t=1}^n x_t$,
      then
      \onslide<4->{
        $\sensitivity{f} = B/n$.
      }
    \end{example}
    \only<article>{
      \begin{proof}
        Consider two neighbouring datasets $x, x'$ differing in example $j$. Then
        \[
        f(x) - f(x')
        = \frac{1}{n}\left[f(x_j) - f(x'_j)\right]
        \leq \frac{1}{n}\left[B - 0\right]
        \]
      \end{proof}
    }
  }
\end{frame}
\begin{frame}
\begin{theorem}
    The Laplace mechanism on a function $f$ with sensitivity $\sensitivity{f}$, ran with $\Laplace(\lambda)$ is $\sensitivity{f} / \lambda$-DP. 
  \end{theorem}
  \begin{proof}
    \begin{align*}
      \frac{\pol(a \mid x)}{\pol(a \mid x')}
      &=
        \frac{e^{|a - f(x')|/\lambda}}{e^{|a - f(x)|/\lambda}}
        \leq
        \frac{e^{|a - f(x)|/\lambda + \sensitivity{f}/\lambda}}{e^{|a - f(x)|/\lambda}}
        = e^{\sensitivity{f} / \lambda}
    \end{align*}
  \end{proof}
  So we need to use $\lambda = \sensitivity{f} / \epsilon$ for $\epsilon$-DP. 
  What is the effect of applying the Laplace mechanism in the local versus centralised model?
  \only<article>{
    Here let us assume $x_i \in [0, B]$ for all $i$ and consider the problem of calculating the average. 
    \begin{block}{Laplace in the local privacy model}
      The sensitivity of the individual data is $B$, so to obtain $\epsilon$-DP we need to use $\lambda = B / \epsilon$. The variance of each component is $2(M/\epsilon)^2$, so the total variance is $2M^2/\epsilon^2 n$.
    \end{block}
    \begin{block}{Laplace in the centralised privacy model}
      The sensitivity of $f$ is $M / n$, so we only need to use $\lambda = \frac{M}{n \epsilon}$. The variance of $a$ is $2(M / \epsilon n)^2$. 
    \end{block}
    Thus the two models have a significant difference in the variance of the estimates obtained, for the same amount of privacy. While the central mechanism has variance $O(n^{-2})$, the local one is $O(n^{-1})$ and so our estimates will need much more data to be accurate under this mechanism. In particular, we need square the amount of data in the local model as we need in the central model. Nevertheless, the local model may be the only possible route if we have no specific use for the data.
  }
\end{frame}

\subsection{Utility of queries}

\begin{frame}
  \only<article>{Rather than saying that we wish to calculate a private version of some specific function $f$, sometimes it is more useful to consider the problem from the perspective of the utility of different answers to queries. More precisely, imagine the interaction between a database system and a user:}
  \begin{block}{Interactive queries}
    \begin{itemize}
    \item System has data $x$.
    \item User asks query $q$.
    \item System responds with $a$.
    \item There is a common utility function
      $\util : \CX, \CA, \CQ \to \Reals$.
    \end{itemize}
    We wish to maximisation $\util$ with our answers, but are constrained by the fact that we also want to preserve privacy.
  \end{block}
  \only<article>{The utility $\util(x,a,q)$  describes how appropriate each response $a$ given by the system for a query $r$ is given the data $x$. It can be seen as how useful the response is~\footnote{This is essentially the utility to the user that asks the query, but it could be the utility to the person that answers. In either case, the motivation does not matter the action should maximise it, but is constrained by privacy.} It allows us to quantify exactly how much we would gain by replying correctly. The exponential mechanism, described below is a simple differentially private mechanism for responding to queries while trying to maximise utility for \alert{any possible} utility function.}

\end{frame}
\begin{frame}
  \frametitle{The Exponential Mechanism.}
  \only<article>{
    Here we assume that we can answer queries $q$, whereby each possible answer $a$ to the query has a different utility to the DM: $\util(q, a, x)$.
    Let $\sensitivity{\util(q)} \defn \sup_{x N x'} |\util(q, a, x) -\util(q, a, x)|$ denote the sensitivity of a query. Then the following mechanism is $\epsilon$-differentially private.
  }
  \begin{definition}[The Exponential mechanism]
    For any utility function $\util : \CQ \times \CA \times \CX \to \Reals$, define the policy
    \begin{equation}
      \label{eq:exponential-mechanism}
      \pol(a \mid x) \defn \frac{e^{\epsilon \util(q, a, x) / \sensitivity{ \util(q)}}}{\sum_{a'} e^{\epsilon \util(q, a', x) / \sensitivity{\util(q)}}}
    \end{equation}
  \end{definition}
  \only<presentation>{
    What happens when $\epsilon \to \infty$? What about when $\epsilon \to 0$?
  }
  \only<article>{
    Clearly, when $\epsilon \to 0$, this mechanism is uniformly random. When $\epsilon \to \infty$ the action maximising $\util(q,a,x)$ is always chosen.

    Although the exponential mechanism can be used to describe most known DP mechanisms, its best use is in settings where there is a natural utility function.
  }
\end{frame}


\subsection{Privacy and reproducibility}

\begin{frame}
  \frametitle{The unfortunate practice of adaptive analysis}
  \begin{tikzpicture}
    \node<1->[rectangle] at (0,4) (prior) {Prior};
    \node<2->[rectangle] at (0,0) (training) {Training data};
    \node<3->[rectangle] at (4,4) (posterior) {Posterior};
    \node<5->[rectangle] at (8,4) (posterior2) {Posterior'};
    \node<2->[rectangle] at (4,0) (holdout) {Holdout};
    \node<4->[RV] at (4,2) (result) {Result};
    \node<5->[RV] at (8,2) (result2) {Result'};
    \draw<3->[medarrow] (training)--(posterior);
    \draw<3->[medarrow] (prior)--(posterior);
    \draw<4->[medarrow] (posterior)--(result);
    \draw<4->[medarrow] (holdout)--(result);
    \draw<5->[red,medarrow] (posterior2)--(result2);
    \draw<5->[red,medarrow] (holdout)--(result2);
    \draw<5->[red,medarrow] (result)--(posterior2);
    \draw<5->[red,medarrow] (posterior)--(posterior2);
  \end{tikzpicture}
  \only<article>{In the ideal data analysis, we start from some prior hypothesis, then obtain some data, which we split into training and holdout. We then examine the training data and obtain a posterior that corresponds to our conclusions. We can then measure the quality of these conclusions in the independent holdout set.

    However, this is not what happens in general. Analysts typically use the same holdout repeatedly, in order to improve the performance of their algorithms. This can be seen as indirectly using the holdout data to obtain a new posterior, and so it is possible that you can overfit on the holdout data, even if you never directly see it. It turns out we can solve this problem if we use differential privacy, so that the analyst only sees a differentially private version of queries.
  }
\end{frame}


\begin{frame}
  \frametitle{The reusable holdout~\cite{dwork2015reusable}\footnote{Also see \url{https://ai.googleblog.com/2015/08/the-reusable-holdout-preserving.html}}}
  \only<article>{One idea to solve this problem is to only allow the analyst to see a private version of the result. In particular, the analyst will only see whether or not the holdout result is $\tau$-close to the training result.}

  \begin{block}{Algorithm parameters}
    \begin{itemize}
    \item Performance measure $f$.
    \item Threshold $\tau$. \only<article>{How close do we want $f$ to be on the training versus holdout set?}
    \item Noise $\sigma$. \only<article>{How much noise should we add?}
    \item Budget $B$. \only<article>{How much are we allowed to learn about the holdout set?}

    \end{itemize}
  \end{block}
  \begin{block}{Algorithm idea}
    \begin{algorithmic}
      \State Run algorithm $\lambda$ on data $\Training$ and get e.g. classifier parameters $\theta$.
      \State Run a DP version of the function $f(\theta, \Holdout) = \ind{\util(\theta, \Training) \geq \tau \util(\theta, \Holdout)}$.
    \end{algorithmic}
  \end{block}
  \only<article>{So instead of reporting the holdout performance at all, you just see if you are much worse than the training performance, i.e. if you're overfitting. The fact that the mechanism is DP also makes it difficult to learn the holdout set. See the thresholdout link for more details.}
\end{frame}

\begin{frame}
  \frametitle{Available privacy toolboxes}
  \begin{block}{$k$-anonymity}
    \begin{itemize}
    \item \url{https://github.com/qiyuangong/Mondrian} Mondrian k-anonymity
    \end{itemize}
  \end{block}
  \begin{block}{Differential privacy}
    \begin{itemize}
    \item \url{https://github.com/bmcmenamin/thresholdOut-explorations}{Threshold out}
    \item \url{https://github.com/steven7woo/Accuracy-First-Differential-Privacy}{Accuracy-constrained DP}
      \item \url{https://github.com/menisadi/pydp}{Various DP algorithms}
\item \url{https://github.com/haiphanNJIT/PrivateDeepLearning} Deep learning and DP
    \end{itemize}
  \end{block}
\end{frame}

\begin{frame}
  \frametitle{Learning outcomes}
  \begin{block}{Understanding}
    \begin{itemize}
    \item Linkage attacks and $k$-anonymity.
    \item Inferring data from summary statistics.
    \item The local versus global differential privacy model.
    \item False discovery rates.
    \end{itemize}
  \end{block}
  
  \begin{block}{Skills}
    \begin{itemize}
    \item Make a dataset satisfy $k$-anonymity with respect to identifying attributes.
    \item Apply the randomised response and Laplace mechanism to data.
    \item Apply the exponential mechanism to simple decision problems.
    \item Use differential privacy to improve reproducibility.
    \end{itemize}
  \end{block}

  \begin{block}{Reflection}
    \begin{itemize}
    \item How can potentially identifying attributes be chosen to achieve $k$-anonymity?
    \item How should the parameters of the two ideas, $\epsilon$-DP and $k$-anonymity be chosen?
    \item Does having more data available make it easier to achieve privacy?
    \end{itemize}
  \end{block}
  
\end{frame}

  
%%% Local Variables:
%%% mode: latex
%%% TeX-engine: xetex
%%% TeX-master: "notes"
%%% End:

 %data bases

\chapter{Fairness}
\only<article>{
  When machine learning algorithms are applied at scale, it can be difficult to imagine
}
\include{fairness}
\section{Project: Credit risk for mortgages}

Consider a bank that must design a decision rule for giving loans to individuals. In this particular case, some of each individual's characteristics are partially known to the bank.  We can assume that the insurer has a linear utility for money and wishes to maximise expected utility. Assume that the $t$-th individual is associated with relevant information $x_t$, sensitive information $z_t$ and a potential outcome $y_t$, which is whether or not they will default on their mortgage. For each individual $t$, the decision rule chooses $a \in \CA$ with probability $\pol(a_t = a \mid x_t)$.

As an example, take a look at the historical data in \texttt{data/credit/german.data-mumeric}, described in \texttt{data/credit/german.doc}. Here there are some attributes related to financial situation, as well as some attributes related to personal information such as gender and marital status. 

A skeleton for the project is available at \url{https://github.com/olethrosdc/ml-society-science/tree/master/src/project-1}. Start with \verb|random_banker.py| as a template, and create a new module \verb|name_banker.py|. You can test your implementation with the \verb|TestLending.py| program. 

For ensuring progress, the project is split into two parts:
\subsection{Deadline 1: September 14}
The first part of the project focuses on a baseline implementation of a banker module.
\begin{enumerate}
\item Design a policy for giving or denying credit to individuals, given their probability for being credit-worthy. Assuming that if an individual is credit-worthy, you will obtain a return on investement of $r = 0.5\%$ per month. Take into account the length of the loan to calculate the utility through \verb|NameBanker.expected_utility()|. Assume that the loan is either fully repaid at the end of the lending period $n$, or not at all to make things simple. If an individual is not credit-worthy you will lose your investment of $m$ credits, otherwise you will gain $m [(1 + r)^{n} - 1]$ . Ignore macroenomic aspects, such as inflation. In this section, simply assume you have a model for predicting creditworthiness as input to your policy, which you can access \verb|NameBanker.get_proba()|. 
\item Implement \verb|NameBanker.fit()| to fit a model for calculating the probability of credit-worthiness from the german data. Then implement \verb|NameBanker.predict_proba()| to predict the probability of the loan being returned for new data. What are the implicit assumptions about the labelling process in the original data, i.e. what do the labels represent?
\item Combine the model with the first policy to obtain a policy for giving credit, given only the information about the individual and previous data seen. In other words, implement \verb|Namebanker.get_best_action()|.
\item Finally, using \verb|TestLending.py| as a baseline, create a jupyter notebook where you document your model development. Then compare your model against \verb|RandomBanker|.
\end{enumerate}

\subsection{Deadline 2: September 28}
The second part of the project focuses on issues of reproducibility, reliability, privacy and fairness. That is, how desirable would it be to use this model in practice? Here are some sample questions that you can explore, but you should be free to think about other questions.
\begin{enumerate}
\item Is it possible to ensure that your policy maximises revenue? How can you take into account the uncertainty due to the limited and/or biased data? What if you have to decide for credit for thousands of individuals and your model is wrong? How should you take that type of risk into account?\footnote{You do not need to implement anything specific for this to pass the assignment, but you should outline an algorithm in a precise enough manner that it can be implemented. In either case you should explain how your solution mitigates this type of risk.}
\item Does the existence of this database raise any privacy concerns? If the database was secret (and only known by the bank), but the credit decisions were public, how would that affect privacy? (a) Explain how you would protect the data of the people in the training set. (b) Explain how would protect the data of the people that apply for new loans. (c) \emph{Implement} a private decision making mechanism for (b),\footnote{If you have already implemented (a) as part of the tutorial, feel free to include the results in your report.} and estimate the amount of loss in utility as you change the privacy guarantee.
\item Choose one concept of fairness, e.g. balance of decisions with respect to gender. How can you ensure that your policy is fair? How can you measure it? How does the original training data affect the fairness of your policy? \footnote{You do not need to implement any type of fair policy a passing grade, but you should at least try to analyse the data or your decision function with simple statistics.}
\end{enumerate}

Submit a final report about your project, either as a standalone PDF or as a jupyter notebook.

%%% Local Variables:
%%% mode: latex
%%% TeX-master: notes
%%% End:




\chapter{Recommendation systems}
\only<article>{Structured learning problems involve multiple latent variables with a complex structure. These range from clustering and spech recognition to DNA and biological and social network analysis. Since structured problems include relationships between many variables, they can be analysed using graphical models.}
\include{recommendation}
\include{clustering}
\include{networks}
\include{hmm}
\include{rnn}

\chapter{Bandit problems}
\include{bandit}
\include{experiment-design}
\chapter{Markov decision processes}
\include{mdp}
\chapter{Safety}
\include{safety}


\bibliographystyle{plainnat}
\bibliography{../bibliography}

\end{document}
%%% Local Variables:
%%% mode: latex
%%% TeX-engine: xetex
%%% TeX-master: "notes"
%%% End:
